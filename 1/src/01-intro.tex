\begin{frame}{Vorstellung}

\begin{itemize}
\itemsep1pt\parskip0pt\parsep0pt
\item
  herzlich willkommen! :)
\item
  Valentin Heinz
  (\href{mailto:Valentin.Heinz@hhu.de}{\nolinkurl{Valentin.Heinz@hhu.de}})
\item
  Computerlinguistik im Master (Abschlussarbeit fehlt noch)
\item
  ich mag \LaTeX \ und Gestaltung
\item
  ich unterrichte und programmiere gerne
\end{itemize}

\end{frame}

\begin{frame}{Praktisches, Fragen}

\begin{itemize}
\itemsep1pt\parskip0pt\parsep0pt
\item
  Sie oder Du?
\item
  haben Sie einen Laptop dabei? Haben Sie Strom?
\item
  wann machen wir Mittagspause?
\end{itemize}

\end{frame}

\begin{frame}{Überblick}

\begin{itemize}
\itemsep1pt\parskip0pt\parsep0pt
\item
  der heutige Inhalt besteht aus folgenden Teilen:

  \begin{itemize}
  \itemsep1pt\parskip0pt\parsep0pt
  \item \LaTeX-Geschichte
  \item \LaTeX-Funktionsweise
  \item \LaTeX-Dokumentstruktur
  \item mehr Befehle
  \item Schriftformatierung
  \item mehr Struktur
  \item Übungen
  \end{itemize}
\item
  wie sieht die Struktur eines Teils aus?

  \begin{itemize}
  \itemsep1pt\parskip0pt\parsep0pt
  \item
    Einstieg: Wiederholung, Fragen
  \item
    Inhalte werden idealer Weise durch den Theorieteil, Praxisteil und Übungen vermittelt
  \end{itemize}
\end{itemize}

\end{frame}

\begin{frame}{Motivation}

\begin{itemize}
\itemsep1pt\parskip0pt\parsep0pt
\item
  oder: warum gebe ich diesen Kurs?
\item
  es gibt viele Übersichten, Bücher, Webseiten usw. zu \LaTeX, aber kaum
  praktische Kurse
\item
  in Seminaren sind die Lernerfolge gut überprüfbar
\item
  die Hilfestellung ist direkt und passend
\item
  Erfahrungsaustausch, Tipps \& Tricks bündeln
\end{itemize}

\end{frame}

\begin{frame}{Vorstellung 2}

\begin{itemize}
\itemsep1pt\parskip0pt\parsep0pt
\item
  Fachsemester
\item
  Vorstellungsrunde:

  \begin{itemize}
  \itemsep1pt\parskip0pt\parsep0pt
  \item
    wie heißen Sie?
  \item
    ihre nächste große Arbeit ist \ldots{} ?
  \item
    welche fachliche Ausrichtung haben Sie?
  \item
    welche Schwerpunkte haben Sie?
  \end{itemize}
\end{itemize}

\end{frame}

\begin{frame}{Dokumenttypen}

\begin{itemize}
\itemsep1pt\parskip0pt\parsep0pt
\item
  Bedarfsevaluation
\item
  welche Dokumenttypen sind für Sie relevant?

  \begin{itemize}
  \itemsep1pt\parskip0pt\parsep0pt
  \item
    Buch
  \item
    Artikel/Hausarbeit
  \item
    Präsentation
  \item
    Magazin
  \item
    Konferenzposter
  \item
    Bewerbungsunterlagen
  \item
    Lebenslauf
  \item
    Umfragebogen
  \item
    Brief
  \item
    Gedicht, Notensatz, Theaterstück, \ldots{}
  \end{itemize}
\end{itemize}

\end{frame}

\begin{frame}{Ziele}

\begin{itemize}
\itemsep1pt\parskip0pt\parsep0pt
\item
  \LaTeX \ lernen und dann \textbf{üben}
\item
  einfacheres, schöneres und effizienteres wissenschaftliches Arbeiten
  mit \LaTeX
\item
  Probleme mit \LaTeX \ selber lösen
\item
  einen Überblick über die Möglichkeiten und das Ökosystem bieten
\item
  Fragen richtig stellen, Hilfe zur Selbsthilfe
\end{itemize}

\end{frame}

\begin{frame}{Eigenleistung}

\begin{itemize}
\itemsep1pt\parskip0pt\parsep0pt
\item
  selbsterstelltes Cheatsheet
\item
  eigenständig erstellte Arbeit, diese beinhaltet:

  \begin{itemize}
  \itemsep1pt\parskip0pt\parsep0pt
  \item
    eine aussagenlogische Formel
  \item
    eine Tabelle mit einer Kopfzeile und mindestens drei normalen Zeilen
  \item
    etwas Fließtext
  \item
    ein Zitat mit Quellenangabe
  \item
    eine besonderheit deines Spezialgebiets (z.B. phonetische
    Transkiption)
  \item
    das Literaturverzeichnis
  \item
    wichtig: Kommentare, welcher Teil des Quelltextes dem jeweiligen
    obigen Teil entspricht
  \end{itemize}
\item
  \scriptsize keine Panik, folgendes sieht nach mehr aus, als es ist! Dennoch stelle
  ich damit sicher, dass Sie mit \LaTeX \ arbeiten können.\normalsize
\end{itemize}

\end{frame}
