\begin{frame}{\LaTeX-Praxis}

\end{frame}

\begin{frame}{Editor}

Um erste \LaTeX-Erfahrungen zu machen

\begin{itemize}
\itemsep1pt\parskip0pt\parsep0pt
\item
  rufen Sie \url{https://www.overleaf.com/} auf
\item
  dort schalten Sie links oben auf \texttt{Source}
\item
  dann schalten Sie rechts oben den \texttt{Preview}-Schalter auf
  \texttt{Manual}
\item
  dann löschen Sie das Dokument
\end{itemize}

\end{frame}

\begin{frame}{erste (!) Dokumenttypen}

Sie müssen bei jedem \LaTeX-Dokument den Dokumenttypen angeben:
\begin{itemize}
\item \texttt{article} Journals, usw.
\item \texttt{report} längere Berichte, kurze Bücher, Diplomarbeiten
\item \texttt{book} richtige Bücher
\item \texttt{letter} Briefe
\item \texttt{beamer} Präsentationen (siehe \LaTeX/Präsentationen)
\end{itemize}

Weiterhin müssen Sie definieren, wo unser Dokument anfängt.

\end{frame}

\begin{frame}[fragile]{Minimalbeispiel (Struktur)}

\begin{verbatim}

\documentclass{article}

\begin{document}
    Hello World!
\end{document}
\end{verbatim}

\begin{itemize}
\item Dokumentaufbau (Nicht-Dokument, Dokument, Text)
\item Frage: was ist was?
\end{itemize}
\end{frame}

\begin{frame}[fragile]{Syntax}

\begin{itemize}
\itemsep1pt\parskip0pt\parsep0pt
\item
    Ab und an werde ich die Befehlssyntax näher besprechen, wie in diesen
    Beispielen
\end{itemize}

\begin{verbatim}
\befehl{PARAMETER}

\begin{PARAMETER}
\end{PARAMETER}
\end{verbatim}

\end{frame}

\begin{frame}[fragile]{Keine Probleme?}

\begin{itemize}
\itemsep1pt\parskip0pt\parsep0pt
\item
  Finden Sie Probleme in diesem Beispiel?
\end{itemize}

\begin{verbatim}
\documentclass{article}
\begin{document}
Außerdem und überhaupt: „wir“ wollen mehr!
\end{document}
\end{verbatim}

\end{frame}

\begin{frame}{Probleme: Umlaute und Sonderzeichen}

\begin{itemize}
\itemsep1pt\parskip0pt\parsep0pt
\item
  Anführungszeichen
\item
  für deutschsprachige Arbeiten unübliche Ränder und Abstände
\item
  keine Kommentare
\end{itemize}

\end{frame}
