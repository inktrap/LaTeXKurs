\begin{frame}{Zusammenfassung der Syntax}

\end{frame}

\begin{frame}{Syntax}

\begin{itemize}
\itemsep1pt\parskip0pt\parsep0pt
\item
  \texttt{PARAMETER} sind Argumente
\item
  \texttt{PARAMETER} sind nicht immer \texttt{OPTIONEN}
\item
  \texttt{OPTIONEN} sind optionale \texttt{PARAMETER}
\item
  unser erster Befehl: \texttt{\textbackslash{}documentclass\{\}}.
  Syntax:
  \texttt{\textbackslash{}documentclass{[}OPTIONEN{]}\{PARAMETER\}}
\item
  unsere erste Umgebung: durch \texttt{\textbackslash{}begin\{\}} und
  \texttt{\textbackslash{}end\{\}}. Syntax:
\end{itemize}

\end{frame}

\begin{frame}[fragile]{Befehlsarten}

\begin{verbatim}
% nicht optional
\befehl{PARAMETER}

% optional
\befehl[OPTIONEN]

% beides
\befehl[OPTIONEN]{PARAMETER}
\end{verbatim}

\end{frame}
