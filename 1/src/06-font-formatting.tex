\begin{frame}[fragile]{Textformatierung}

\begin{itemize}
\itemsep1pt\parskip0pt\parsep0pt
\item
  Die Textformatierung findet mit einem Schalter statt.
\item
  Dieser nimmt keine Parameter und bleibt bei Größenangaben aktiv, bis
  ein anderer Schalter aktiv wird.
\item
  Hinweis: \texttt{\textbackslash{}LaTeX}, etc. ist kein Schalter,
  sondern nur ein Textbefehl
\end{itemize}

Syntax:

\begin{verbatim}
% schalter ist nicht aktiv
\schaltername
% schalter ist aktiv
\end{verbatim}

\end{frame}

\begin{frame}{Größe}

\begin{itemize}
\itemsep1pt\parskip0pt\parsep0pt
\item
  \texttt{\textbackslash{}tiny}
\item
  \texttt{\textbackslash{}scriptsize}
\item
  \texttt{\textbackslash{}footnotesize}
\item
  \texttt{\textbackslash{}small}
\item
  \texttt{\textbackslash{}normalsize}
\item
  \texttt{\textbackslash{}large}
\item
  \texttt{\textbackslash{}Large}
\item
  \texttt{\textbackslash{}LARGE}
\item
  \texttt{\textbackslash{}huge}
\item
  \texttt{\textbackslash{}Huge}
\end{itemize}

\end{frame}

\begin{frame}{Übung}

\begin{itemize}
\itemsep1pt\parskip0pt\parsep0pt
\item
  betrachten Sie folgende Geschichte:
\item
  \texttt{tiny\ and\ huge\ are\ walking\ to\ the\ small\ house\ of\ footnotesize.\ tiny\ says:\ “this\ is\ too\ Huge\ for\ me”\ and\ huge\ thinks\ thats\ not\ normalsize\ either}
\item
  formatieren Sie das Wort der ersten Größe mit der letzten vorkommenden
  Größe.
\item
  d.h.: tiny mit normalsize, huge mit huge, usw.
\end{itemize}

\end{frame}

\begin{frame}{Größe als Umgebung}

\begin{itemize}
\itemsep1pt\parskip0pt\parsep0pt
\item
  die Größe kann auch per Umgebung definiert werden:
  \texttt{\textbackslash{}begin\{scriptsize\}\ Ein\ kleiner\ Satz.\ \textbackslash{}end\{scriptsize\}\ Ich\ bin\ wieder\ normal.}
\item
  dies ist häufig viel leichter zu überschauen, da man die Umgebung
  bewusst beendet.
\end{itemize}

\end{frame}

\begin{frame}[fragile]{Schriftformatierung und Übung}

Übernehmen Sie folgende Formatierungsangaben in Ihr Cheatsheet:

\begin{verbatim}
Laut
\textsc{Gaius Iulius Caesar} sollte man neue Begriffe
\textit{kursiv} schreiben und nicht
\textbf{fett}, besonders
\texttt{Schreibmaschinen} sind eher
\texttt{\textbf{\textit{total}}} unnötig.
\end{verbatim}

\end{frame}

\begin{frame}{Hausaufgabe: Farben}

\begin{itemize}
\itemsep1pt\parskip0pt\parsep0pt
\item
  finden Sie heraus, wie sie Text farbig gestalten können
\item
  zeigen Sie dies morgen an einem Beispiel
\end{itemize}

\end{frame}
