\begin{frame}[fragile]{Ein zweiter Blick auf Überschriften}
    \begin{itemize}[<+->]
        \item \lstinline| \setcounter{secnumdepth}{1}|
        \item setzt die Variable secnumdepth auf 1
        \item 1 steht für \texttt{section} (umgebungsabhängig)
        \item 2 steht für \texttt{subsection} usw.
        \item setzen Sie den Zähler auf 1 und schauen Sie, wie sich die Überschriften verhalten
    \end{itemize}
\end{frame}


\begin{frame}[fragile]{Ein zweiter Blick auf Zähler}
    \begin{itemize}[<+->]
        \item Syntax: \lstinline|  \setcounter{COUNTER}{INT}|
        \item alle nummerierten Elemente haben einen Zähler
        \item wie setzt man den Zähler der aktuelle Überschrift (\texttt{section}) auf 100?
    \end{itemize}
\end{frame}


\begin{frame}[fragile]{Ein zweiter Blick auf Zähler}
    \begin{itemize}[<+->]
        \item \lstinline| \setcounter{secnumdepth}{2}|
        \item setzt die Variable \texttt{secnumdepth} auf 2
        \item was macht der Befehl: \lstinline|\addtocounter|?
    \end{itemize}
\end{frame}

