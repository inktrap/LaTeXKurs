\documentclass[]{beamer}
% ignorenonframetext
\usetheme{Dresden}
\setbeamertemplate{caption}[numbered]
\setbeamertemplate{caption label separator}{:}
\setbeamercolor{caption name}{fg=normal text.fg}
\usepackage{amssymb,amsmath}
\usepackage{ifxetex,ifluatex}
\usepackage{lmodern}
\usepackage{metalogo}
\usepackage{listings}
\usepackage{hyperref}
\lstset{
    language=[LaTeX]{TeX},
      basicstyle=\normalsize\ttfamily,        % size:normal, family:mono
      breaklines=true,
    }

\ifxetex
  \usepackage{fontspec,xltxtra,xunicode}
  \defaultfontfeatures{Mapping=tex-text,Scale=MatchLowercase}
  \newcommand{\euro}{€}
\else
  \ifluatex
    \usepackage{fontspec}
    \defaultfontfeatures{Mapping=tex-text,Scale=MatchLowercase}
    \newcommand{\euro}{€}
  \else
    \usepackage[T1]{fontenc}
    \usepackage[ngerman]{babel}
    \usepackage[utf8]{inputenc}
      \fi
\fi
% use upquote if available, for straight quotes in verbatim environments
\IfFileExists{upquote.sty}{\usepackage{upquote}}{}
% use microtype if available
\IfFileExists{microtype.sty}{\usepackage{microtype}}{}

% Comment these out if you don't want a slide with just the
% part/section/subsection/subsubsection title:
\AtBeginPart{
  \let\insertpartnumber\relax
  \let\partname\relax
  \frame{\partpage}
}
\AtBeginSection{
  \let\insertsectionnumber\relax
  \let\sectionname\relax
  \frame{\sectionpage}
}
\AtBeginSubsection{
  \let\insertsubsectionnumber\relax
  \let\subsectionname\relax
  \frame{\subsectionpage}
}

\setlength{\parindent}{0pt}
\setlength{\parskip}{6pt plus 2pt minus 1pt}
\setlength{\emergencystretch}{3em}  % prevent overfull lines
\setcounter{secnumdepth}{0}

\date{}
\author{Valentin Heinz}
\title{\LaTeX \ für Einsteiger}
\subtitle{Zweite Sitzung}
\date{\today}

%\includeonly{}

\begin{document}

\maketitle

\title{Wiederholung}
\begin{frame}{Wiederholung}
\begin{itemize}[<+->]
    \item Dokumentstruktur
    \item Dokumenttypen
    \item Überschriften
    \item Textformatierung (Stil, Größe, Farbe) (apropos Farbe)
    \item Blockzitate
    \item Befehle und deren Syntax
    \item Fehler finden
\end{itemize}
\end{frame}

\title{Fragen von letzter Stunde}
% small{} wie \small?
% \subsriptsize?
% ellipsis-Effekt?

\begin{frame}[fragile]{Fragen}
    \begin{itemize}
          \item \lstinline| \tiny{}| hat die selbe Wirkung wie \lstinline| \tiny|, letztere Variante ist empfehlenswert
          \item \texttt{ellipsis} sind integriert
          \item \lstinline|\subscriptsize| ist kein Befehl zur Änderung der Schriftgröße. Subskripte sind sowas: $a_{1}$
          \item beliebige Buchstabengrößen können per \lstinline|\fontsize{INT}{INT}| und \lstinline|\selectfont| verwendet werden (schriftabhängig!)
    \end{itemize}
\end{frame}



\title{Themen}



\title{Dokumenttypen und Schrift}
\begin{frame}[fragile]{Fonts}

\begin{itemize}
      \item für \LaTeX wurde von Knuth die Schriftfamilie \textit{Computer Modern} entworfen
      \item CM ist open-source
      \item Latin Modern ist der Nachfolger von CM (\lstinline|usepackage{lmodern}|)
      \item eine andere „schöne“ romanische Schrift ist \texttt{palatino}
      \item die Schriftart sollte nach der Lesbarkeit ausgewählt werden
      \item Schriften werden nur mit Grund gewechselt
      \item häufige Schriftwechsel sind schlechter Stil
\end{itemize}
\end{frame}

\begin{frame}[fragile]{Familien}
    \begin{itemize}
      \item roman, sans-serif, typewriter
      \item \lstinline| \rmdefault{}|: setzt eine romanische Schrift als default-roman
      \item \lstinline| \sfdefault{}|: setzt eine serifenlose Schrift als default-sans
      \item \lstinline| \ttdefault{}|: setzt eine monospace Schrift als default-mono
      \item folgende Kommandos greifen auf die default-Werte zurück:
      \item \lstinline|\rmfamily{}|, \lstinline|\textsf{}|, \lstinline|\texttt{}|
    \end{itemize}
\end{frame}

\begin{frame}[fragile]{Übung}
    \begin{itemize}
        \item was ist \texttt{PSNFSS} und was ist das \lstinline|textcomp|-Paket?
        \item verwenden Sie Helvetica
        \item verwenden Sie New Century Schoolbook (wie der Supreme Court!)
        \item verwenden Sie Courier als monospace-Schrift
    \end{itemize}
\end{frame}

\begin{frame}[fragile]{Textausrichtung}
    \begin{itemize}
        \item benutzen Sie die Umgebungen
        \item \texttt{flushleft}
        \item \texttt{flushright} und
        \item \texttt{center}
    \end{itemize}
\end{frame}

\begin{frame}[fragile]{unformatierter Text}
    \begin{itemize}
        \item benutzen Sie die \texttt{verbatim} Umgebung um Text unformatiert darzustellen
        \item wie funktioniert der inline-Befehl \lstinline|verb|?
        \item was ist hier anders als bei vorherigen Befehlen?
        \item was ist der Unterschied zu \lstinline|\texttt{}|?
    \end{itemize}
\end{frame}

\title{Titelseite}

\begin{frame}[fragile]{Die Titelseite}
    \begin{itemize}
        \item \lstinline| \title{Der Titel}|
        \item \lstinline| \author{Ich habe das geschrieben}|
        \item \lstinline| \date{\today}|
        \item durch die Spracheinstellungen stimmt das Datumsformat
        \item erstellt die Titelseite: \lstinline| \maketitle|
        \item mehr \href{https://github.com/inktrap/LaTeXKurs/blob/master/2/tex/01-titlepage.tex}{auf GitHub} (via golatex)
    \end{itemize}
\end{frame}

\title{Reservierte Zeichen}


\begin{frame}[fragile]{Bedeutungstragende Zeichen}
    \begin{itemize}
        \item folgende Zeichen sind in \LaTeX bedeutungstragend:
        \item \texttt{\textbackslash \textasciicircum{} \textasciitilde{}}
        \item \lstinline|_ ^ # & $ % { }|
        \item diese müssen deshalb maskiert bzw. durch einen Befehl ausgedrückt werden
    \end{itemize}
\end{frame}

\begin{frame}[fragile]{Übungen}
    \begin{itemize}
        \item warum kann der Backslash nicht maskiert werden?
        \item was ergibt ein maskierter Backslash?
        \item was müssen Sie eingeben, um \texttt{\textbackslash \textasciicircum{} \textasciitilde{}} darzustellen?
        \item maskieren Sie \lstinline|_ ^ # & $ % { }|
        \item wie stellen wir deutsche Anführungszeichen dar?
    \end{itemize}
\end{frame}

%\texttt{\textbackslash, \textasciicircum{}, \textasciitilde{}}
%\texttt{\_ \^ \# \& \$ \%}


\title{Leerzeichen und Abstände}
\begin{frame}{Wortabstände}
\begin{itemize}
    \item \texttt{\textbackslash} erzeugt Leerstelle
    \item \texttt{\textbackslash@} nachfolgender Punkt ist Satzende
    \item \texttt{\textasciitilde}  nicht umbrechbare Leerstelle 
    \item \texttt{\textbackslash{,}} verkürzte nicht umbrechbare Leerstelle
    \item \texttt{{\textbackslash quad}} vergrößerter Abstand
    \item \texttt{{\textbackslash qquad}} vierfach vergrößerter Abstand
    \item \texttt{{\textbackslash hspace\{1cm}\}} 1cm Abstand
\end{itemize}
\end{frame}

\begin{frame}[fragile]{Zeilenabstände}
    \begin{itemize}
        \item Paket: \lstinline|setspace|
        \item recherchieren Sie zu die verfügbaren Optionen
        \item setzen Sie die Standardzeilengröße auf \texttt{double}
        \item verwenden Sie für einen Teil den einfachen Zeilenabstand
        \item verwenden Sie auch den eineinhalbfachen Zeilenabstand
    \end{itemize}
\end{frame}


\begin{frame}[fragile]{horizontal und vertikal}
    \begin{itemize}
        \item benutzen Sie \lstinline|\hspace{}| und \lstinline|\vspace{}|
        \item was machen \lstinline|\smallskip| und \lstinline|\medskip| und \lstinline|\bigskip|?
    \end{itemize}
\end{frame}



\title{Seitengestaltung}
\begin{frame}[fragile]{Zeilen definieren}
    \begin{itemize}
        \item \lstinline|\pagestyle{}| (in der Präambel) und \lstinline|\thispagestyle{}| (im Dokument)
        \item \lstinline|\pagenumbering{}| (in der Präambel)
        \item falls \texttt{myheadings} zu unflexibel ist: \texttt{scrpage2} und \texttt{fancyhdr}
    \end{itemize}
\end{frame}



\title{Nummerierung anpassen}
\begin{frame}[fragile]{Ein zweiter Blick auf Überschriften}
    \begin{itemize}[<+->]
        \item \lstinline| \setcounter{secnumdepth}{1}|
        \item setzt die Variable secnumdepth auf 1
        \item 1 steht für \texttt{section} (umgebungsabhängig)
        \item 2 steht für \texttt{subsection} usw.
        \item setzen Sie den Zähler auf 1 und schauen Sie, wie sich die Überschriften verhalten
    \end{itemize}
\end{frame}


\begin{frame}[fragile]{Ein zweiter Blick auf Zähler}
    \begin{itemize}[<+->]
        \item Syntax: \lstinline|  \setcounter{COUNTER}{INT}|
        \item alle nummerierten Elemente haben einen Zähler
        \item wie setzt man den Zähler der aktuelle Überschrift (\texttt{section}) auf 100?
    \end{itemize}
\end{frame}


\begin{frame}[fragile]{Ein zweiter Blick auf Zähler}
    \begin{itemize}[<+->]
        \item \lstinline| \setcounter{secnumdepth}{2}|
        \item setzt die Variable \texttt{secnumdepth} auf 2
        \item was macht der Befehl: \lstinline|\addtocounter|?
    \end{itemize}
\end{frame}



\title{Eigene Befehle}
\begin{frame}[fragile]{Eigene Befehle definieren}
    \begin{itemize}
        \item eigene Befehle erstellen mit \lstinline|\newcommand{}{}|
        \item dies geschieht nicht im Dokument!
        \item \lstinline|  \newcommand{\zB}{z.\,B. }|
        \item \lstinline|  \newcommand{\hallo}[1]{Hallo: #1}|
        \item es können maximal 9 Parameter übergeben werden (Argumentnennung muss stimmen)
        \item Hinweis: eigene Umgebungen und Längen sind auch möglich
    \end{itemize}
\end{frame}


\title{Hausaufgabe}
\begin{frame}{HA}
    
\begin{itemize}
    \item HA: Texlive und IDE installieren
    \item HA2: falls HA1 geschafft wurde, oder während der Download läuft, lesen Sie: \href{http://www2.informatik.hu-berlin.de/sv/lehre/typographie.pdf}{Einige typographische Grundregeln und ihre Umsetzung in \LaTeX} von Werner Struckmann
    \end{itemize}
\end{frame}


\title{Die IDE}
\begin{frame}[fragile]{TeXLive}
    \begin{itemize}
        \itemsep1pt\parskip0pt\parsep0pt
        \item laden Sie eine LaTeX-Distribution herunter
        \item hier entscheiden wir uns für texlive
        \item Windows: runterladen, installieren (4GB freier Speicher!) \url{http://mirror.ctan.org/systems/texlive/tlnet/install-tl-windows.exe}
        \item Anleitung: \url{https://tug.org/texlive/windows.html}
    \end{itemize}
\end{frame}

\begin{frame}[fragile]{TeXstudio}
    \begin{itemize}
        \item Windows: runterladen, installieren:
        \url{http://www.texstudio.org/}
        \item open-source, syntax-highlighting, konfigurierbar, usw.
        \item \texttt{Optionen} dann \texttt{TeXstudio einrichten} dann \texttt{erweiterte Optionen} anklicken
        \item \texttt{erweiterte Einstellungen}
        \begin{itemize}
            \item Zeilennummern anschalten
            \item Whitespaces anzeigen
            \item Tab zu Space
            \item Einrückung per Spaces
        \end{itemize}
    \end{itemize}
\end{frame}


\title{Pause}
\begin{frame}{Pause}

\Huge Pause?

\end{frame}


\end{document}
