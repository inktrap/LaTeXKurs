\begin{frame}[fragile]{TeXLive}
    \begin{itemize}
        \itemsep1pt\parskip0pt\parsep0pt
        \item laden Sie eine LaTeX-Distribution herunter
        \item hier entscheiden wir uns für texlive
        \item Windows: runterladen, installieren (4GB freier Speicher!) \url{http://mirror.ctan.org/systems/texlive/tlnet/install-tl-windows.exe}
        \item Anleitung: \url{https://tug.org/texlive/windows.html}
    \end{itemize}
\end{frame}

\begin{frame}[fragile]{TeXstudio}
    \begin{itemize}
        \item Windows: runterladen, installieren:
        \url{http://www.texstudio.org/}
        \item open-source, syntax-highlighting, konfigurierbar, usw.
        \item \texttt{Optionen} dann \texttt{TeXstudio einrichten} dann \texttt{erweiterte Optionen} anklicken
        \item \texttt{erweiterte Einstellungen}
        \begin{itemize}
            \item Zeilennummern anschalten
            \item Whitespaces anzeigen
            \item Tab zu Space
            \item Einrückung per Spaces
        \end{itemize}
    \end{itemize}
\end{frame}
