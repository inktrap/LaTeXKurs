\begin{frame}{Themen}
\begin{itemize}
    \item Optionen für Dokumenttypen, Umgebungen
    \item Schrift, Schriftformen
    \item reservierte Zeichen, \lstinline|verbatim|-Umgebung, \lstinline|verb| usw.
    \item Leerzeichen, Wortabstände
    \item Kopf- und Fußzeilen, Seitenzahlen
    \item Nummerierungen
    \item eigene Befehle
    \item eine Tabelle

\end{itemize}
\end{frame}

\begin{frame}{Optionen für den Dokumenttypen}
    \begin{itemize}
        \item verwenden Sie die Optionen: \lstinline| 12pt,twoside,paper=a4,ngerman |
        \item verwenden Sie ab sofort \lstinline|oneside| statt \lstinline|twoside|
        \item verändern Sie probeweise die Papierformate
    \end{itemize}
\end{frame}

\begin{frame}{Weitere Strukturelemente}
Verwenden Sie jedes Strukturelement mindestens einmal:
    \begin{itemize}
        \item \lstinline| section|
        \item \lstinline| subsection|
        \item \lstinline| subsubsection|
        \item \lstinline| paragraph|
        \item \lstinline| subparagraph|
    \end{itemize}
\end{frame}
