\begin{frame}[fragile]{float-Umgebung}
    \begin{itemize}[<+->]
        \item float machen wir spontan, das kennen wir auch schon von \texttt{table}
        \item Grafiken einbinden (Optionen bitte nachlesen, das sind recht viele):
    \end{itemize}
    \begin{lstlisting}
    \usepackage{graphicx}
    % Achtung, richtige Stelle, Olga!
    \includegraphics[Optionen]{Pfad/Dateiname}
    \end{lstlisting}
\end{frame}

\begin{frame}[fragile]{HA? HA!}
    \begin{itemize}[<+->]
    \item Hausaufgabe: versuchen Sie\footnote{Versuchen Sie es wirklich, das sind nur ein paar Zeilen, die anders sind!}, eines ihrer Dokumente unter \XeTeX zu erstellen
    \item Hausaufgabe 2: spielen Sie mit \lstinline|label| und dem Paket \lstinline|varioref| herum
    \item Hausaufgabe 3: seien Sie stark und inkludieren Sie alles einmal testweise!
    \end{itemize}
\end{frame}

\begin{frame}[fragile]{Ende}
    \begin{itemize}[<+->]
    \item damit wären wir auch schon wieder am Ende
    \item nun brauchen wir nur noch Fußnoten und ein Literaturverzeichnis
    \item und dann kommen die Matheumgebung und die Linguistikpakete!
    \item gut gemacht! Bis morgen!
    \end{itemize}
\end{frame}