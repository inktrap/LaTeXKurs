
\begin{frame}[fragile]{Hausputz}
    \begin{itemize}[<+->]
        \item die IDE gibt uns neue Möglichkeiten!
        \item räumen wir unser Template auf, schaffen wir Übersicht!
        \item wir erstellen einen Ordner \texttt{article}
        \item wir verschieben unser Template in den Ordner \texttt{article} und nennen es \texttt{template.tex}
        \item wir nehmen alles zwischen \lstinline|\begin{document}| und \lstinline|\end{document}| heraus und speichern es in \texttt{document.tex}
        \item wir binden die neue Datei per \lstinline|\begin{frame}{Struktur und Text}
    \begin{itemize}[<+->]
        \item Dokumenterstellung, Dokumentklasse, Optionen
        \item Sprachoptionen, Eingabeencoding, Buchstabenencoding
        \item Abschnitte und Unterabschnitte
        \item Blindtextpaket laden und Text einfügen
    \end{itemize}
\end{frame}

\begin{frame}{Wiederholung}
    \begin{itemize}[<+->]
        \item fontenc: Ausgabeformatierung (passende Ausgabe erzeugen) \textbf{zuerst laden!}
        \item babel: Sprachoptionen laden, z.B. Datumsanpassung an den Sprachraum
        \item inputenc: Eingabeformatierung (viele Zeichen/Buchstaben direkt eingeben) \textbf{danach laden!}
        \item zur Kompilierung benutzen wir \texttt{pdfTeX}
        \item aber es gibt auch \XeTeX und \LuaTeX
    \end{itemize}
\end{frame}

\begin{frame}[fragile]{Schriftformatierung}
    \begin{itemize}
        \item \href{https://tex.stackexchange.com/questions/59403/what-font-packages-are-installed-in-tex-live/59405#59405}{stackexchange} hilft weiter
        \item für helvetica einfach \lstinline|\usepackage{helvet}| verwenden. Fertig.
        \item einfach \lstinline|\usepackage{newcnt}| verwenden. Fertig.
        \item nicht-Blindtext kursiv, monospace, fett bzw. als Kapitälchen formatieren
        \item die vorherigen Formatierungen mischen
        \item Serifenformatierung und Romanische Schrift für ausgewählte Abschnitte benutzen
        \item \lstinline|\rmfamily{}| und \lstinline|\sffamily|
    \end{itemize}
\end{frame}

\begin{frame}{Ausrichtung, Absätze, Lass-mich-in-Ruhe}
    \begin{itemize}
        \item wie formatieren wir einen Teil rechtsbündig, Blocksatz bzw. linksbündig?
        \item wie erzeugen wir einen Zeilenumbruch?
        \item wie einen Paragraphen?
        \item wie unformatierten Text (Umgebung, inline)?
    \end{itemize}
\end{frame}



| in \texttt{template.tex} in das Dokument ein
        \item Puh! Das war viel! Geht alles?
    \end{itemize}
\end{frame}

\begin{frame}[fragile]{Einbinden}
    \begin{itemize}[<+->]
        \item wir steuern in unserem Template per \lstinline| \includeonly{} | die Einbindung
        \item dies macht unsere Übungen einfacher, da wir nun \texttt{document} deaktivieren können
        \item nun erstellen wir \texttt{titlepage.txt} und inkludieren \texttt{titlepage}
        \item auch in \lstinline| \includeonly{} |
    \end{itemize}
\end{frame}