\begin{frame}{Webseite}
\begin{itemize}[<+->]
    \item Seminarseite: \href{https://github.com/inktrap/LaTeXKurs}{https://github.com/inktrap/LaTeXKurs}
    \item Folien, Wiki, Listen, Quellcode, Bilder, usw.
    \item GitHub bietet eine Versionsgeschichte und noch viel mehr
    \item kann von den Teilnehmern kritisiert werden: Issue/Bugreport
    \item kann erweitert werden: Pull-Request, Wikiaccount
    \item st{und diese Folien hier sind auch schon online!}
    \item aber: häufige Änderungen, da viel passiert. Status: unstable
\end{itemize}
\end{frame}

\begin{frame}{Ziele der heutigen Sitzung}
    \begin{itemize}[<+->]
    \item Ziel: alle wichtigen Konzepte in einem Dokument
    \item Methode: wir programmieren gemeinsam nach einem strukturierten Ablauf
    \item Definitionen und derzeitigen Schritt per Folie
    \item Interaktion: ähnlich zum Pair Programming
    \item Ergebnis: alle haben den selben Stand, das selbe Template
    \item Schrittumsetzung: Teilnehmer
    \end{itemize}
\end{frame}

\begin{frame}{TeXstudio}
    \begin{itemize}
        \item Vervollständigung: für den Lernprozess schlecht, für die Produktivität gut
        \item konnten Sie die IDE und texlive/miktex installieren?
        \item richten Sie TeXstudio/TeXmaker ein, Bilder: \href{https://github.com/inktrap/LaTeXKurs/tree/master/3/img}{https://github.com/inktrap/LaTeXKurs/tree/master/3/img}
    \end{itemize}
\end{frame}

\begin{frame}{TeXstudio}
    \begin{itemize}
        \item Vervollständigung: für den Lernprozess schlecht, für die Produktivität gut
        \item richten Sie TeXstudio/TeXmaker ein, Bilder:
        \item \href{https://github.com/inktrap/LaTeXKurs/tree/master/3/img}{https://github.com/inktrap/LaTeXKurs/tree/master/3/img}
    \end{itemize}
\end{frame}