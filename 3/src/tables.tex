\begin{frame}[fragile]{Tabellen}
    \begin{itemize}[<+->]
        \item eine Tabelle hat einen Anfang: \lstinline|\begin{tabular}|
        \item und ein Ende \lstinline|\end{tabular}|
        \item dann kommt eine Kopfzeile: \lstinline| Pferde & Schafe & Ziegen & Hasen \\|
        \item dann eben Zeilen mit Werten: \lstinline| 1 & 2 & 3 & 4 \\ |
        \item jetzt wissen wir auch warum wir \& maskieren müssen.
    \end{itemize}
\end{frame}

\begin{frame}[fragile]{Tabellen mit Strichen}
    \begin{itemize}[<+->]
        \item einen vertikalen Strich fügen wir mit \lstinline|\hline| hinter \lstinline|\\| hinzu
        \item die Zellenausrichtung bestimmen wir vorher \lstinline|\begin{tabular}{lrlr}|
        \item möglich ist auch \texttt{c}. Wichtig: Pro Spalte eine Angabe, maximal!
        \item was passiert bei weniger/mehr?
        \item was macht \lstinline!\begin{tabular}{l|r|l|r}|!?
    \end{itemize}
\end{frame}

\begin{frame}[fragile]{table-Umgebung}
    wir legen unsere Tabelle (\texttt{tabular}) in eine \texttt{table}-Umgebung
        \begin{lstlisting}
        \begin{table}[h!]
            \begin{center}
                \caption{Caption for the table.}
                \label{tab:table1}
                % tabelle hier hin
            \end{center}
        \end{table}
        \end{lstlisting}
\end{frame}

\begin{frame}[fragile]{Neu gelernt!}
    \begin{itemize}[<+->]
        \item \lstinline|\table|
        \item Platzierung ist wichtig!
        \item Tabellenplatzierung \lstinline|\begin{table}[Position] ... \end{table}|
        \item Parameter: \lstinline|b, h, p und t (Default: tbp)| (Kombination mit \texttt{!} erzwingt!)
        \item \lstinline|\centering|
        \item \lstinline|\caption|
        \item \lstinline|\label| und \lstinline|\ref|? 
    \end{itemize}
\end{frame}