\documentclass[]{beamer}
% ignorenonframetext
\usetheme{Dresden}
\setbeamertemplate{caption}[numbered]
\setbeamertemplate{caption label separator}{:}
\setbeamercolor{caption name}{fg=normal text.fg}
\usepackage{amssymb,amsmath}
\usepackage{textcomp}
\usepackage{ifxetex,ifluatex}
\usepackage{lmodern}
\usepackage{metalogo}
\usepackage{hyperref}
\usepackage{metalogo} % if you want to add LaTeX within your code
\usepackage{listings}
\usepackage{soul}

\ifxetex
  \usepackage{fontspec,xltxtra,xunicode}
  \defaultfontfeatures{Mapping=tex-text,Scale=MatchLowercase}
  \newcommand{\euro}{€}
\else
  \ifluatex
    \usepackage{fontspec}
    \defaultfontfeatures{Mapping=tex-text,Scale=MatchLowercase}
    \newcommand{\euro}{€}
  \else
    \usepackage[T1]{fontenc}
    \usepackage[ngerman]{babel}
    \usepackage[utf8]{inputenc}
      \fi
\fi
% use upquote if available, for straight quotes in verbatim environments
\IfFileExists{upquote.sty}{\usepackage{upquote}}{}
% use microtype if available
\IfFileExists{microtype.sty}{\usepackage{microtype}}{}

% Comment these out if you don't want a slide with just the
% part/section/subsection/subsubsection title:
\AtBeginPart{
  \let\insertpartnumber\relax
  \let\partname\relax
  \frame{\partpage}
}
\AtBeginSection{
  \let\insertsectionnumber\relax
  \let\sectionname\relax
  \frame{\sectionpage}
}
\AtBeginSubsection{
  \let\insertsubsectionnumber\relax
  \let\subsectionname\relax
  \frame{\subsectionpage}
}

\setlength{\parindent}{0pt}
\setlength{\parskip}{6pt plus 2pt minus 1pt}
\setlength{\emergencystretch}{3em}  % prevent overfull lines
\setcounter{secnumdepth}{0}

\lstset{
    language=[LaTeX]{TeX},
      basicstyle=\normalsize\ttfamily,        % size:normal, family:mono
      breaklines=true,
    }

\date{}
\author{Valentin Heinz}
\title{\LaTeX \ für Einsteiger}
\subtitle{Dritte Sitzung}
\date{\today}

\begin{document}

\maketitle

\title{Ablauf}
\begin{frame}[fragile]{Ablauf}
    \begin{itemize}[<+->]
        \item Hausaufgabenbesprechung
        \item \LaTeX\ ohne Mathe? Nein: Mathematik/Formeln!
        \item Phonetik: IPA mit \texttt{tipa}
        \item Alinierungen, Beispiele, usw.: \texttt{covington}
        \item Zeichnen mit \texttt{tikz}
        \item Bäume: \texttt{forest}
        \item Merkmalsstrukturen: \texttt{avm}
    \end{itemize}
\end{frame}


\title{Grundgerüst}
\begin{frame}{Struktur und Text}
    \begin{itemize}[<+->]
        \item Dokumenterstellung, Dokumentklasse, Optionen
        \item Sprachoptionen, Eingabeencoding, Buchstabenencoding
        \item Abschnitte und Unterabschnitte
        \item Blindtextpaket laden und Text einfügen
    \end{itemize}
\end{frame}

\begin{frame}{Wiederholung}
    \begin{itemize}[<+->]
        \item fontenc: Ausgabeformatierung (passende Ausgabe erzeugen) \textbf{zuerst laden!}
        \item babel: Sprachoptionen laden, z.B. Datumsanpassung an den Sprachraum
        \item inputenc: Eingabeformatierung (viele Zeichen/Buchstaben direkt eingeben) \textbf{danach laden!}
        \item zur Kompilierung benutzen wir \texttt{pdfTeX}
        \item aber es gibt auch \XeTeX und \LuaTeX
    \end{itemize}
\end{frame}

\begin{frame}[fragile]{Schriftformatierung}
    \begin{itemize}
        \item \href{https://tex.stackexchange.com/questions/59403/what-font-packages-are-installed-in-tex-live/59405#59405}{stackexchange} hilft weiter
        \item für helvetica einfach \lstinline|\usepackage{helvet}| verwenden. Fertig.
        \item einfach \lstinline|\usepackage{newcnt}| verwenden. Fertig.
        \item nicht-Blindtext kursiv, monospace, fett bzw. als Kapitälchen formatieren
        \item die vorherigen Formatierungen mischen
        \item Serifenformatierung und Romanische Schrift für ausgewählte Abschnitte benutzen
        \item \lstinline|\rmfamily{}| und \lstinline|\sffamily|
    \end{itemize}
\end{frame}

\begin{frame}{Ausrichtung, Absätze, Lass-mich-in-Ruhe}
    \begin{itemize}
        \item wie formatieren wir einen Teil rechtsbündig, Blocksatz bzw. linksbündig?
        \item wie erzeugen wir einen Zeilenumbruch?
        \item wie einen Paragraphen?
        \item wie unformatierten Text (Umgebung, inline)?
    \end{itemize}
\end{frame}





\title{include}

\begin{frame}[fragile]{Hausputz}
    \begin{itemize}[<+->]
        \item die IDE gibt uns neue Möglichkeiten!
        \item räumen wir unser Template auf, schaffen wir Übersicht!
        \item wir erstellen einen Ordner \texttt{article}
        \item wir verschieben unser Template in den Ordner \texttt{article} und nennen es \texttt{template.tex}
        \item wir nehmen alles zwischen \lstinline|\begin{document}| und \lstinline|\end{document}| heraus und speichern es in \texttt{document.tex}
        \item wir binden die neue Datei per \lstinline|\begin{frame}{Struktur und Text}
    \begin{itemize}[<+->]
        \item Dokumenterstellung, Dokumentklasse, Optionen
        \item Sprachoptionen, Eingabeencoding, Buchstabenencoding
        \item Abschnitte und Unterabschnitte
        \item Blindtextpaket laden und Text einfügen
    \end{itemize}
\end{frame}

\begin{frame}{Wiederholung}
    \begin{itemize}[<+->]
        \item fontenc: Ausgabeformatierung (passende Ausgabe erzeugen) \textbf{zuerst laden!}
        \item babel: Sprachoptionen laden, z.B. Datumsanpassung an den Sprachraum
        \item inputenc: Eingabeformatierung (viele Zeichen/Buchstaben direkt eingeben) \textbf{danach laden!}
        \item zur Kompilierung benutzen wir \texttt{pdfTeX}
        \item aber es gibt auch \XeTeX und \LuaTeX
    \end{itemize}
\end{frame}

\begin{frame}[fragile]{Schriftformatierung}
    \begin{itemize}
        \item \href{https://tex.stackexchange.com/questions/59403/what-font-packages-are-installed-in-tex-live/59405#59405}{stackexchange} hilft weiter
        \item für helvetica einfach \lstinline|\usepackage{helvet}| verwenden. Fertig.
        \item einfach \lstinline|\usepackage{newcnt}| verwenden. Fertig.
        \item nicht-Blindtext kursiv, monospace, fett bzw. als Kapitälchen formatieren
        \item die vorherigen Formatierungen mischen
        \item Serifenformatierung und Romanische Schrift für ausgewählte Abschnitte benutzen
        \item \lstinline|\rmfamily{}| und \lstinline|\sffamily|
    \end{itemize}
\end{frame}

\begin{frame}{Ausrichtung, Absätze, Lass-mich-in-Ruhe}
    \begin{itemize}
        \item wie formatieren wir einen Teil rechtsbündig, Blocksatz bzw. linksbündig?
        \item wie erzeugen wir einen Zeilenumbruch?
        \item wie einen Paragraphen?
        \item wie unformatierten Text (Umgebung, inline)?
    \end{itemize}
\end{frame}



| in \texttt{template.tex} in das Dokument ein
        \item Puh! Das war viel! Geht alles?
    \end{itemize}
\end{frame}

\begin{frame}[fragile]{Einbinden}
    \begin{itemize}[<+->]
        \item wir steuern in unserem Template per \lstinline| \includeonly{} | die Einbindung
        \item dies macht unsere Übungen einfacher, da wir nun \texttt{document} deaktivieren können
        \item nun erstellen wir \texttt{titlepage.txt} und inkludieren \texttt{titlepage}
        \item auch in \lstinline| \includeonly{} |
    \end{itemize}
\end{frame}

\title{Titelseite}
\title{Titel der Arbeit}
\subtitle{Ein Untertitel}
\author{Autor der Arbeit}
\date{\today}
\maketitle
\begin{abstract}
    Abstract: Das ist ein kleines abstract, aber wohl nicht an der richtigen Stelle, mit \LaTeX
\end{abstract}
\thispagestyle{empty}
\clearpage
\tableofcontents

\title{Seitenstil}
\begin{frame}[fragile]{Zeilenformatierung}
    \begin{itemize}
        \item um in den Genuss von Kopf- bzw. Fußzeilen zu kommen benutzen wir: \lstinline|\pagestyle{ARGUMENT}|
        \item wobei \texttt{ARGUMENT} eines aus: \texttt{plain}, \texttt{empty}, \texttt{headings} oder \texttt{myheadings} sein muss.
        \item probieren wir, vllt. mit etwas Blindtext die Unterschiede aus
        \item \texttt{myheadings} benötigt
        \item \lstinline|\markboth{links}{rechts \thepage}|
        \item oder \lstinline|\markright{ich stehe rechts}|.
        \item Hinweis: die Seitenzahlen können wir mit \lstinline|\pagenumbering{NUMSTYLE}| ändern
        \item \texttt{NUMSTYLE} ist hier z.B. \texttt{arabic}, oder \texttt{roman, alph, Alpha, Roman}
    \end{itemize}
\end{frame}

\title{Tabellen}
\begin{frame}[fragile]{Tabellen}
    \begin{itemize}[<+->]
        \item eine Tabelle hat einen Anfang: \lstinline|\begin{tabular}|
        \item und ein Ende \lstinline|\end{tabular}|
        \item dann kommt eine Kopfzeile: \lstinline| Pferde & Schafe & Ziegen & Hasen \\|
        \item dann eben Zeilen mit Werten: \lstinline| 1 & 2 & 3 & 4 \\ |
        \item jetzt wissen wir auch warum wir \& maskieren müssen.
    \end{itemize}
\end{frame}

\begin{frame}[fragile]{Tabellen mit Strichen}
    \begin{itemize}[<+->]
        \item einen vertikalen Strich fügen wir mit \lstinline|\hline| hinter \lstinline|\\| hinzu
        \item die Zellenausrichtung bestimmen wir vorher \lstinline|\begin{tabular}{lrlr}|
        \item möglich ist auch \texttt{c}. Wichtig: Pro Spalte eine Angabe, maximal!
        \item was passiert bei weniger/mehr?
        \item was macht \lstinline!\begin{tabular}{l|r|l|r}|!?
    \end{itemize}
\end{frame}

\begin{frame}[fragile]{table-Umgebung}
    wir legen unsere Tabelle (\texttt{tabular}) in eine \texttt{table}-Umgebung
        \begin{lstlisting}
        \begin{table}[h!]
            \begin{center}
                \caption{Caption for the table.}
                \label{tab:table1}
                % tabelle hier hin
            \end{center}
        \end{table}
        \end{lstlisting}
\end{frame}

\begin{frame}[fragile]{Neu gelernt!}
    \begin{itemize}[<+->]
        \item \lstinline|\table|
        \item Platzierung ist wichtig!
        \item Tabellenplatzierung \lstinline|\begin{table}[Position] ... \end{table}|
        \item Parameter: \lstinline|b, h, p und t (Default: tbp)| (Kombination mit \texttt{!} erzwingt!)
        \item \lstinline|\centering|
        \item \lstinline|\caption|
        \item \lstinline|\label| und \lstinline|\ref|? 
    \end{itemize}
\end{frame}

\title{Float und Grafiken}
\begin{frame}[fragile]{Grafiken}
    \begin{itemize}[<+->]
        \item das Paket \texttt{graphicx} laden
        \item kann Windows den \texttt{/}?
    \end{itemize}
    \begin{lstlisting}
    \usepackage{graphicx}
    % Achtung, richtige Stelle, Olga!
    \includegraphics[Optionen]{Pfad/Dateiname}
    \end{lstlisting}
\end{frame}

\begin{frame}[fragile]{Etablierte Einbindung von Grafiken}
    \begin{itemize}[<+->]
        \item im Rahmen unserer Projektplanung wählen wir einen relativen Pfad zum Bildordner
        \item Linux: \texttt{../../img/knuth.jpg} unter Windows: ?
    \end{itemize}
        \begin{lstlisting}
        \includegraphics[
            width=0.8\textwidth,
            height=0.8\textheight,
            keepaspectratio]
        {PFAD}
        \end{lstlisting}
\end{frame}

\begin{frame}[fragile]{Etablierte Einbindung von Grafiken}
    \begin{itemize}[<+->]
        \item darum legen wir nun eine \texttt{figure}-Umgebung
        \item \texttt{figure} ist so ähnlich wie \texttt{table}
        \item dort können wir auch wieder vieles bestimmen
        \item Beispiele: \href{https://en.wikibooks.org/wiki/LaTeX/Floats,_Figures_and_Captions}{Wikibook}
    \end{itemize}
\end{frame}


\title{Hausaufgaben}

\begin{frame}[fragile]{Fragen?}
    \begin{itemize}[<+->]
    \item wie immer: habt ihr Fragen?
    \item Noch immer?
    \item Nein?
    \item damit wären wir auch schon wieder am Ende!
    \item vielen Dank und viel Erfolg!
    \end{itemize}
\end{frame}




%\title{Pause}

\end{document}
