\documentclass[12pt,oneside,paper=a4,ngerman]{scrartcl}

\usepackage[T1]{fontenc}
\usepackage[ngerman]{babel}
% utf8x kann mehr sonderzeichen usw.
% wird aber wohl nicht mehr gepflegt und
% utf8 ist die bessere Wahl. Oder XeTeX
\usepackage[utf8]{inputenc}

\usepackage{blindtext}
\usepackage{newcent}

\begin{document}
    % hier üben wir überschriften
    \section{Hallo Welt}
    \subsection{Oh Gott, nein dafür bin ich nicht so kreativ!}
    \blindtext[1]
    \section{Weil es so toll ist}
    \subsection{Ein weiterer Punkt}
    Hallo
    % FIXME warum funktioniert das nicht? 
    %\begin{textit}
    %    Valentin
    %\end{textit}
    
    %\textit{Valentin}
    %FIXME: inputenc-Probleme
    
    % hier üben wir textformatierungen
    \textsc{Valentin} muss \textit{newcnt} zu newcent korrigieren in den Folien
    weil niemand einen \textbf{GitHub}-\texttt{Account} hat
    \textbf{\texttt{geschachtelt}}. \rmfamily{romanisch} und \sffamily. Unterschied. \textsf{Das hier geht auch}. Wirklich
     
     % hier wird die textgrösse geändert
    \tiny Wirklich klein
    \normalsize Normal
    \tiny{Wirklich klein}
    % FIXME aber schalter sind doch besser, weil der Skopus von \tiny{skopus} auch das hier umfasst.
    
    % reset der Größe und Schriftfamilie
    \normalsize
    \rmfamily
    
    % Hier formatieren wir Text rechtsseitig, linksseitig
    % und als zentriert
    \begin{flushright}
    \blindtext
    \end{flushright}
    \begin{center}
    \blindtext
    \end{center}
    \begin{flushleft}
    \blindtext
    \end{flushleft}
     
    % harter Zeilenumbruch
    ich bin eine Zeile\\
    und ich bin eine neuen Zeile
    
    % diese Schreibweise ist konsistent zur
    % Überschriftenschreibweise
    \paragraph{Ich bin ein Absatz \blindtext}
        Das hier ist wieder normale Schrift.
    \subparagraph{Unterabsatz \blindtext}
        Das hier ist wieder normale Schrift.
    
    % aber das geht auch als Umgebung
    \begin{paragraph}
        Ein neuer Absatz als Umgebung \blindtext
    \end{paragraph}

    Das hier ist wieder normale Schrift.
    
    % das hier ist Text völlig unformatiert
    \begin{verbatim}
    ich erscheine einfach so
    \textbf{ich bin gar nicht so dick}
    Zeilenumbrüche gehen auch direkt
    \end{verbatim}
    
    Ich bin ein normale Umgebung, hier geht
    ein Zeilenumbruch ganz anders.
    
    % unformatiert, inline
    Jetzt noch \verb+inline alles was man schreiben will+ \\
    Neu; \verb_hier ist das mit einem Unterstrich_ \\
    
    \begin{verbatim}
    \verb+Ein plus in dieser Zeile\++
    dafür kann man aber jedes  mögliche Zeichen
    statt dem Plus nehmen
    \end{verbatim}

    % von 10:00 - 10:50 Installationsschmerzen
    % von 10:50 - 12:30 sind wir bis "Hausputz" gekommen
    % Mittagspause bis 13:30
\end{document}






