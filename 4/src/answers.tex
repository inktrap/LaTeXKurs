\begin{frame}[fragile]{Hausaufgaben}

\begin{itemize}[<+->]
    \item Hausaufgabe: versuchen Sie\footnote{Das war doch schwieriger als gedacht!}, eines ihrer Dokumente unter \XeTeX zu erstellen
    \item Lösung: \href{https://github.com/inktrap/LaTeXKurs/tree/master/4/tex/}{XeTeX-Beispiel}
    \item Hausaufgabe 2: spielen Sie mit \lstinline|label| und \lstinline|ref| herum
    \item Lösung: \lstinline|\label{labelname}| und \lstinline|ref{labelname}| danach und mehrfach kompilieren
    \item Hausaufgabe 2.5: machen Sie sich auch mit dem Paket \lstinline|varioref| vertraut, wo liegen hier die Unterschiede zu \lstinline|ref|?
    \item \href{https://github.com/inktrap/LaTeXKurs/tree/master/4/tex/}{Beispiel}
\end{itemize}
    
\end{frame}

\begin{frame}[fragile]{Hausaufgaben}
    
    \begin{itemize}[<+->]
        \item Hausaufgabe 3: seien Sie stark und inkludieren Sie alles einmal testweise!
        \item dauert lange, oder?
        \item Hausaufgabe 4: setzen Sie den Zähler nach dem Deckblatt zurück
        \item Lösung: Seitenzähler zurücksetzen: \lstinline|\setcounter{page}{1}| oder \lstinline|\addtocounter{page}{-1}|
        \item Hausaufgabe 5: setzen Sie die Art der Seitenzahl der aktuellen Seite auf \texttt{roman}
        \item \begin{lstlisting}
        \pagenumbering{roman}
        foobar
        \clearpage
        \pagenumbering{arabic}
        foobar
        \end{lstlisting}
    \end{itemize}

\end{frame}

