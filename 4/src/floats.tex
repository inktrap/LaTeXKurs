\begin{frame}[fragile]{Grafiken}
    \begin{itemize}[<+->]
        \item das Paket \texttt{graphicx} laden
    \end{itemize}
    \begin{lstlisting}
    \usepackage{graphicx}
    % Achtung, richtige Stelle, Olga!
    \includegraphics[Optionen]{Pfad/Dateiname}
    \end{lstlisting}
\end{frame}

\begin{frame}[fragile]{Etablierte Einbindung von Grafiken}
    \begin{itemize}[<+->]
        \item im Rahmen unserer Projektplanung wählen wir einen relativen Pfad zum Bildordner
        \item Linux: \texttt{../../img/knuth.jpg} (auch unter Windows getestet)
    \end{itemize}
        \begin{lstlisting}
        \includegraphics[
            width=0.8\textwidth,
            height=0.8\textheight,
            keepaspectratio]
        {PFAD}
        \end{lstlisting}
\end{frame}

\begin{frame}[fragile]{Etablierte Einbindung von Grafiken}
    \begin{itemize}[<+->]
        \item darum legen wir nun eine \texttt{figure}-Umgebung
        \item \texttt{figure} ist so ähnlich wie \texttt{table}
        \item dort können wir auch wieder vieles bestimmen
        \item Beispiele: \href{https://en.wikibooks.org/wiki/LaTeX/Floats,_Figures_and_Captions}{Wikibook}
    \end{itemize}
\end{frame}
