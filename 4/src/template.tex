\documentclass[]{beamer}
% ignorenonframetext
\usetheme{Dresden}
\setbeamertemplate{caption}[numbered]
\setbeamertemplate{caption label separator}{:}
\setbeamercolor{caption name}{fg=normal text.fg}
\usepackage{amssymb,amsmath}
\usepackage{textcomp}
\usepackage{ifxetex,ifluatex}
\usepackage{lmodern}
\usepackage{metalogo}
\usepackage{hyperref}
\usepackage{metalogo} % if you want to add LaTeX within your code
\usepackage{listings}
\usepackage{soul}

\ifxetex
  \usepackage{fontspec,xltxtra,xunicode}
  \defaultfontfeatures{Mapping=tex-text,Scale=MatchLowercase}
  \newcommand{\euro}{€}
\else
  \ifluatex
    \usepackage{fontspec}
    \defaultfontfeatures{Mapping=tex-text,Scale=MatchLowercase}
    \newcommand{\euro}{€}
  \else
    \usepackage[T1]{fontenc}
    \usepackage[ngerman]{babel}
    \usepackage[utf8]{inputenc}
      \fi
\fi
% use upquote if available, for straight quotes in verbatim environments
\IfFileExists{upquote.sty}{\usepackage{upquote}}{}
% use microtype if available
\IfFileExists{microtype.sty}{\usepackage{microtype}}{}

% Comment these out if you don't want a slide with just the
% part/section/subsection/subsubsection title:
\AtBeginPart{
  \let\insertpartnumber\relax
  \let\partname\relax
  \frame{\partpage}
}
\AtBeginSection{
  \let\insertsectionnumber\relax
  \let\sectionname\relax
  \frame{\sectionpage}
}
\AtBeginSubsection{
  \let\insertsubsectionnumber\relax
  \let\subsectionname\relax
  \frame{\subsectionpage}
}

\setlength{\parindent}{0pt}
\setlength{\parskip}{6pt plus 2pt minus 1pt}
\setlength{\emergencystretch}{3em}  % prevent overfull lines
\setcounter{secnumdepth}{0}

\lstset{
    language=[LaTeX]{TeX},
      basicstyle=\normalsize\ttfamily,        % the size of the fonts that are used for the code
    }

\date{}
\author{Valentin Heinz}
\title{\LaTeX \ für Einsteiger}
\subtitle{Vierte Sitzung}
\date{\today}

%\includeonly{beamer, biber}

\begin{document}

\maketitle

\title{Ablauf}
\begin{frame}[fragile]{Ablauf}
    \begin{itemize}[<+->]
        \item Hausaufgabenbesprechung
        \item \LaTeX\ ohne Mathe? Nein: Mathematik/Formeln!
        \item Phonetik: IPA mit \texttt{tipa}
        \item Alinierungen, Beispiele, usw.: \texttt{covington}
        \item Zeichnen mit \texttt{tikz}
        \item Bäume: \texttt{forest}
        \item Merkmalsstrukturen: \texttt{avm}
    \end{itemize}
\end{frame}


\title{Hausaufgaben}
\begin{frame}[fragile]{Hausaufgaben}

\begin{itemize}[<+->]
    \item Hausaufgabe: versuchen Sie\footnote{Das war doch schwieriger als gedacht!}, eines ihrer Dokumente unter \XeTeX zu erstellen
    \item Lösung: \href{https://github.com/inktrap/LaTeXKurs/tree/master/4/tex/}{XeTeX-Beispiel}
    \item Hausaufgabe 2: spielen Sie mit \lstinline|label| und \lstinline|ref| herum
    \item Lösung: \lstinline|\label{labelname}| und \lstinline|ref{labelname}| danach und mehrfach kompilieren
    \item Hausaufgabe 2.5: machen Sie sich auch mit dem Paket \lstinline|varioref| vertraut, wo liegen hier die Unterschiede zu \lstinline|ref|?
    \item \href{https://github.com/inktrap/LaTeXKurs/tree/master/4/tex/}{Beispiel}
\end{itemize}
    
\end{frame}

\begin{frame}[fragile]{Hausaufgaben}
    
    \begin{itemize}[<+->]
        \item Hausaufgabe 3: seien Sie stark und inkludieren Sie alles einmal testweise!
        \item dauert lange, oder?
        \item Hausaufgabe 4: setzen Sie den Zähler nach dem Deckblatt zurück
        \item Lösung: Seitenzähler zurücksetzen: \lstinline|\setcounter{page}{1}| oder \lstinline|\addtocounter{page}{-1}|
        \item Hausaufgabe 5: setzen Sie die Art der Seitenzahl der aktuellen Seite auf \texttt{roman}
        \item \begin{lstlisting}
        \pagenumbering{roman}
        foobar
        \clearpage
        \pagenumbering{arabic}
        foobar
        \end{lstlisting}
    \end{itemize}

\end{frame}



\title{Fehlende Teile}
\begin{frame}[fragile]{Lücken füllen}

\begin{itemize}[<+->]
    \item Abstracts sind auch nur Umgebungen im Dokument
    \item der Anker\lstinline|\footnote{hier bin ich}| bildet Fußnoten\footnote{hier bin ich}. Da geht aber noch mehr\footnote{bigfoot and manyfoot}. Wie man sieht\footnote{Werden Fußnoten seitenübergreifend hochgezählt}
    \item deutsche Anführungszeichen, entweder: Unicode »sowas« oder „sowas“
    \item oder \lstinline|"` bzw. \glqq| oder \lstinline|"' or \grqq|, aber Spaß macht das nicht. \glqq{}Test\grqq{}\footnote{Vergesst die \texttt{\{\}} nicht}
    \item fallen euch noch Sachen ein, die wir machen sollten?
\end{itemize}

\end{frame}

\title{Booktabs}
\begin{frame}[fragile]{Booktabs}

\begin{itemize}[<+->]
    \item publikationsgerechte Tabellen
    \item keine waagerechten Linien, keine Doppellinien
    \item das \texttt{booktabs}-Paket laden
    \item \href{ftp://ftp.rrzn.uni-hannover.de/pub/mirror/tex-archive/macros/latex/contrib/booktabs-de/booktabs-de.pdf}{Booktabs-Beispiele}
    \item \lstinline|\toprule \midrule \bottomrule|
\end{itemize}

\end{frame}

\title{Float und Grafiken}
\begin{frame}[fragile]{Grafiken}
    \begin{itemize}[<+->]
        \item das Paket \texttt{graphicx} laden
        \item kann Windows den \texttt{/}?
    \end{itemize}
    \begin{lstlisting}
    \usepackage{graphicx}
    % Achtung, richtige Stelle, Olga!
    \includegraphics[Optionen]{Pfad/Dateiname}
    \end{lstlisting}
\end{frame}

\begin{frame}[fragile]{Etablierte Einbindung von Grafiken}
    \begin{itemize}[<+->]
        \item im Rahmen unserer Projektplanung wählen wir einen relativen Pfad zum Bildordner
        \item Linux: \texttt{../../img/knuth.jpg} unter Windows: ?
    \end{itemize}
        \begin{lstlisting}
        \includegraphics[
            width=0.8\textwidth,
            height=0.8\textheight,
            keepaspectratio]
        {PFAD}
        \end{lstlisting}
\end{frame}

\begin{frame}[fragile]{Etablierte Einbindung von Grafiken}
    \begin{itemize}[<+->]
        \item darum legen wir nun eine \texttt{figure}-Umgebung
        \item \texttt{figure} ist so ähnlich wie \texttt{table}
        \item dort können wir auch wieder vieles bestimmen
        \item Beispiele: \href{https://en.wikibooks.org/wiki/LaTeX/Floats,_Figures_and_Captions}{Wikibook}
    \end{itemize}
\end{frame}


%\title{Mathematik}
%\begin{frame}[fragile]{Mathematik}
    \begin{itemize}[<+->]
        \item Achtung: eine kurze, unvollständige Einführung!!
        \item Bereiche: im Text oder in einer Umgebung
        \item \texttt{inline}: \lstinline|\( Mathematikmodus \) $ Mathematikmodus $ |
        \item oder als Umgebung. Wie?
        \item längere Formeln kann man als \texttt{displaymath} setzen:
        \item \lstinline|\[ \]| oder \lstinline|\begin{displaymath} \end{displaymath}|
        \item Gleichungen per \texttt{equation}-Umgebung
        \item \texttt{amsmath}-Paket ist der Standard
    \end{itemize}
\end{frame}

\begin{frame}[fragile]{Mathematik: Brüche, Wurzel}
    \begin{itemize}[<+->]
        \item Syntax: \lstinline|\frac{numerator}{denominator}|
        \item \lstinline|\(\frac{a}{b}\)|\is \(\frac{a}{b}\)
        \item Syntax: \lstinline|base^{exponent}| Bsp: \item \lstinline|3^{20}|\is \(3^{20}\)
        \item \lstinline| \( 5 + 3^{2} = 14\)|\is{} \( 5 + 3^{2} = 14\)
        \item \lstinline|\(\sqrt{\frac{a}{b}}\)|\is \(\sqrt{\frac{a}{b}}\)
        \item Indices: \(a_{1}\)
    \end{itemize}
\end{frame}

\begin{frame}[fragile]{Beispiele}
    \begin{itemize}[<+->]
        \item \lstinline| \(\neg\forall x P(x) \equiv \exists x \neg P(x)\)|\is \(\neg\forall x P(x) \equiv \exists x \neg P(x)\)
        \item \lstinline| \( \frac{1}{n} \sum_{i=i}^{n} x_{i} \)|\is{} \( \frac{1}{n} \sum_{i=i}^{n} x_{i} \)
    \end{itemize}
    \begin{lstlisting}
\begin{displaymath}
    \cos A \cos B
    = \frac{1}{2}\left[ \cos(A-B)+\cos(A+B) \right]
\end{displaymath}
\end{lstlisting}
\begin{displaymath}\cos A \cos B
    = \frac{1}{2}\left[ \cos(A-B)+\cos(A+B) \right]
\end{displaymath}
\end{frame}

\begin{frame}[fragile]{Aligned numbered Equation}
    \begin{lstlisting}
    \begin{align}
    z_0 &= d = 0 \\
    z_{n+1} &= z_n^2+c
    \end{align}
    \end{lstlisting}
    \begin{align}
    z_0 &= d = 0 \\
    z_{n+1} &= z_n^2+c
    \end{align}
\end{frame}



\begin{frame}[fragile]{Zusammenfassung}
    \begin{itemize}[<+->]
        \item Leerzeichen werden nicht interpretiert
        \item \texttt{\textbackslash} per Makro \lstinline|\backslash|
        \item \lstinline|^| und \lstinline|_| haben eine besondere Bedeutung
        \item es gibt ein paar Symbole: \href{https://en.wikibooks.org/wiki/LaTeX/Mathematics#List_of_Mathematical_Symbols}{per default}
        \item \href{http://milde.users.sourceforge.net/LUCR/Math/mathpackages/amssymb-symbols.pdf}{Amssymb hat mehr}
        \item Skoping ist wichtig: \lstinline|2^ab| ist nicht $ 2^{ab} $ sondern $ 2^ab$
        \item Bonus: \href{http://detexify.kirelabs.org/classify.html}{Symbolerkennung}
        \item Aufgabe: setzen Sie eine Formel für das arithmetische Mittel per Summenformel
    \end{itemize}
\end{frame}



\title{Bibliographie}
\documentclass[12pt,oneside,paper=a4,ngerman]{scrartcl}

\usepackage[T1]{fontenc}
\usepackage[ngerman]{babel}
% utf8x kann mehr sonderzeichen usw.
% wird aber wohl nicht mehr gepflegt und
% utf8 ist die bessere Wahl. Oder XeTeX
\usepackage[utf8]{inputenc}

\usepackage{blindtext}
\usepackage{newcent}

\usepackage{hyperref}
\usepackage{csquotes}

\usepackage[backend=biber,
        bibstyle=biblatex-sp-unified,
        citestyle=sp-authoryear-comp,
        maxcitenames=3,
        maxbibnames=99]{biblatex}

% danach die bibliographie laden, .bib ist obligatorisch!
\addbibresource{example.bib}

\begin{document}
\section{Ich bin ein Beispiel}

We have a sentence \cite[12]{Knuth:1997:ACP:260999}

According to this guy \citet[1]{Knuth:1997:ACP:260999} is the following stuff true:
\begin{quote}
    \blindtext
\end{quote}
According to this guy \citep[1]{Knuth:1997:ACP:260999}:
\begin{quote}
    \blindtext
\end{quote}
\clearpage
\printbibliography
\end{document}





\title{Präsentationen}
\documentclass[]{beamer}
\usetheme{Berlin}

% unsere Pakete wie immer
\usepackage[T1]{fontenc}
\usepackage[ngerman]{babel}
% utf8x kann mehr sonderzeichen usw.
% wird aber wohl nicht mehr gepflegt und
% utf8 ist die bessere Wahl. Oder XeTeX
\usepackage[utf8]{inputenc}

\author{Der gesamte Kurs}
\title{Hello Kurs}
\date{24.12.2015}

% frame darf man NIE einrücken und es darf auch
% KEIN Kommentar nach dem Öffnen der Frame-Umgebung kommen!!
\begin{document}

\maketitle

\begin{frame}[fragile]{Ääähm Grundkurs Linguistik}
    \begin{itemize}
    \item Umgebung Frame aufmachen. Maaaaann. HAHAHAA.
    \end{itemize}
\end{frame}

% wir lassen items nach und nach erscheinen

\begin{frame}[fragile]{Ääähm Grundkurs Linguistik}
    \begin{itemize}[<+->]
        \item Umgebung Frame aufmachen.
        \item aber nach und nach
        \item Maaaaann. HAHAHAA.
    \end{itemize}
\end{frame}

% viele andere Optionen und Templates gibt es auch

\end{document}
%\title{Pause}

\title{Hausaufgaben}

\begin{frame}[fragile]{Fragen?}
    \begin{itemize}[<+->]
    \item wie immer: habt ihr Fragen?
    \item Noch immer?
    \item Nein?
    \item damit wären wir auch schon wieder am Ende!
    \item vielen Dank und viel Erfolg!
    \end{itemize}
\end{frame}




\end{document}
