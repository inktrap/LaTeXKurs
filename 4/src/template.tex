\documentclass[]{beamer}
% ignorenonframetext
\usetheme{Dresden}
\setbeamertemplate{caption}[numbered]
\setbeamertemplate{caption label separator}{:}
\setbeamercolor{caption name}{fg=normal text.fg}
\usepackage{amssymb,amsmath}
\usepackage{textcomp}
\usepackage{ifxetex,ifluatex}
\usepackage{lmodern}
\usepackage{metalogo}
\usepackage{hyperref}
\usepackage{metalogo} % if you want to add LaTeX within your code
\usepackage{listings}
\usepackage{soul}

\ifxetex
  \usepackage{fontspec,xltxtra,xunicode}
  \defaultfontfeatures{Mapping=tex-text,Scale=MatchLowercase}
  \newcommand{\euro}{€}
\else
  \ifluatex
    \usepackage{fontspec}
    \defaultfontfeatures{Mapping=tex-text,Scale=MatchLowercase}
    \newcommand{\euro}{€}
  \else
    \usepackage[T1]{fontenc}
    \usepackage[ngerman]{babel}
    \usepackage[utf8]{inputenc}
      \fi
\fi
% use upquote if available, for straight quotes in verbatim environments
\IfFileExists{upquote.sty}{\usepackage{upquote}}{}
% use microtype if available
\IfFileExists{microtype.sty}{\usepackage{microtype}}{}

% Comment these out if you don't want a slide with just the
% part/section/subsection/subsubsection title:
\AtBeginPart{
  \let\insertpartnumber\relax
  \let\partname\relax
  \frame{\partpage}
}
\AtBeginSection{
  \let\insertsectionnumber\relax
  \let\sectionname\relax
  \frame{\sectionpage}
}
\AtBeginSubsection{
  \let\insertsubsectionnumber\relax
  \let\subsectionname\relax
  \frame{\subsectionpage}
}

\setlength{\parindent}{0pt}
\setlength{\parskip}{6pt plus 2pt minus 1pt}
\setlength{\emergencystretch}{3em}  % prevent overfull lines
\setcounter{secnumdepth}{0}

\lstset{
    language=[LaTeX]{TeX},
      basicstyle=\normalsize\ttfamily,        % the size of the fonts that are used for the code
    }

\date{}
\author{Valentin Heinz}
\title{\LaTeX \ für Einsteiger}
\subtitle{Vierte Sitzung}
\date{\today}

%\includeonly{beamer, biber}

\begin{document}

\maketitle

\title{Ablauf}
\begin{frame}[fragile]{Ablauf}
    \begin{itemize}[<+->]
        \item Hausaufgabenbesprechung
        \item \LaTeX\ ohne Mathe?
        \item Mathematik/Formeln
        \item Phonetik: IPA mit \texttt{tipa}
        \item Alinierungen, Beispiele, usw.: \texttt{covington}
        \item Zeichnen mit \texttt{tikz}
        \item Bäume: \texttt{forest}, \texttt{avm}
    \end{itemize}
\end{frame}


\title{Hausaufgaben}
\begin{frame}[fragile]{Hausaufgaben}

\begin{itemize}[<+->]
    \item Hausaufgabe: versuchen Sie\footnote{Das war doch schwieriger als gedacht!}, eines ihrer Dokumente unter \XeTeX zu erstellen
    \item Lösung: \href{https://github.com/inktrap/LaTeXKurs/tree/master/4/tex/}{XeTeX-Beispiel}
    \item Hausaufgabe 2: spielen Sie mit \lstinline|label| und \lstinline|ref| herum
    \item Lösung: \lstinline|\label{labelname}| und \lstinline|ref{labelname}| danach und mehrfach kompilieren
    \item Hausaufgabe 2.5: machen Sie sich auch mit dem Paket \lstinline|varioref| vertraut, wo liegen hier die Unterschiede zu \lstinline|ref|?
    \item \href{https://github.com/inktrap/LaTeXKurs/tree/master/4/tex/}{Beispiel}
\end{itemize}
    
\end{frame}

\begin{frame}[fragile]{Hausaufgaben}
    
    \begin{itemize}[<+->]
        \item Hausaufgabe 3: seien Sie stark und inkludieren Sie alles einmal testweise!
        \item dauert lange, oder?
        \item Hausaufgabe 4: setzen Sie den Zähler nach dem Deckblatt zurück
        \item Lösung: Seitenzähler zurücksetzen: \lstinline|\setcounter{page}{1}| oder \lstinline|\addtocounter{page}{-1}|
        \item Hausaufgabe 5: setzen Sie die Art der Seitenzahl der aktuellen Seite auf \texttt{roman}
        \item \begin{lstlisting}
        \pagenumbering{roman}
        foobar
        \clearpage
        \pagenumbering{arabic}
        foobar
        \end{lstlisting}
    \end{itemize}

\end{frame}



\title{Fehlende Teile}
\begin{frame}[fragile]{Lücken füllen}

\begin{itemize}[<+->]
    \item Abstracts sind auch nur Umgebungen im Dokument
    \item der Anker\lstinline|\footnote{hier bin ich}| bildet Fußnoten\footnote{hier bin ich}. Da geht aber noch mehr\footnote{bigfoot and manyfoot}. Wie man sieht\footnote{Werden Fußnoten seitenübergreifend hochgezählt}
    \item deutsche Anführungszeichen, entweder: Unicode »sowas« oder „sowas“
    \item oder \lstinline|"` bzw. \glqq| oder \lstinline|"' or \grqq|, aber Spaß macht das nicht. \glqq{}Test\grqq{}\footnote{Vergesst die \texttt{\{\}} nicht}
    \item fallen euch noch Sachen ein, die wir machen sollten?
\end{itemize}

\end{frame}

\title{Booktabs}
\begin{frame}[fragile]{Booktabs}

\begin{itemize}[<+->]
    \item publikationsgerechte Tabellen
    \item keine waagerechten Linien, keine Doppellinien
    \item das \texttt{booktabs}-Paket laden
    \item \href{ftp://ftp.rrzn.uni-hannover.de/pub/mirror/tex-archive/macros/latex/contrib/booktabs-de/booktabs-de.pdf}{Booktabs-Beispiele}
    \item \lstinline|\toprule \midrule \bottomrule|
\end{itemize}

\end{frame}

\title{Float und Grafiken}
\begin{frame}[fragile]{float-Umgebung}
    \begin{itemize}[<+->]
        \item float machen wir spontan, das kennen wir auch schon von \texttt{table}
        \item Grafiken einbinden (Optionen bitte nachlesen, das sind recht viele):
    \end{itemize}
    \begin{lstlisting}
    \usepackage{graphicx}
    % Achtung, richtige Stelle, Olga!
    \includegraphics[Optionen]{Pfad/Dateiname}
    \end{lstlisting}
\end{frame}

\begin{frame}[fragile]{HA? HA!}
    \begin{itemize}[<+->]
    \item Hausaufgabe: versuchen Sie\footnote{Versuchen Sie es wirklich, das sind nur ein paar Zeilen, die anders sind!}, eines ihrer Dokumente unter \XeTeX zu erstellen
    \item Hausaufgabe 2: spielen Sie mit \lstinline|label| und dem Paket \lstinline|varioref| herum
    \item Hausaufgabe 3: seien Sie stark und inkludieren Sie alles einmal testweise!
    \end{itemize}
\end{frame}

\begin{frame}[fragile]{Ende}
    \begin{itemize}[<+->]
    \item damit wären wir auch schon wieder am Ende
    \item nun brauchen wir nur noch Fußnoten und ein Literaturverzeichnis
    \item und dann kommen die Matheumgebung und die Linguistikpakete!
    \item gut gemacht! Bis morgen!
    \end{itemize}
\end{frame}

%\title{Mathematik}
%\section{test}

\begin{displaymath}
\neg \forall x P(x) \equiv \exists x \neg P(x)
\end{displaymath}

\begin{align}
z_0 &= d = 0 \\
z_{n+1} &= z_n^2+c
\end{align}

\begin{displaymath}
    \sum_{i}^{a}
\end{displaymath}


\title{Bibliographie}
\begin{frame}[fragile]{Bibliographie}
    
    \begin{itemize}[<+->]
        \item Dank an Timm für den Tipp mit dem \href{http://celxj.org/downloads/USS-NoComments.pdf}{unified-Stil}
        \item ich habe da mal was vorbereitet: \href{}{Bibliographie mit Biber und unified}
        \item wenn das geht: mit Zotero oder Jabref die Bibliographie verwalten
    \end{itemize}
\end{frame}

\title{Präsentationen}
\begin{frame}[fragile]{Beamer}
    \begin{itemize}[<+->]
        \item Dokumentklasse und Theme
    \end{itemize}
\begin{lstlisting}
    % handouts durch:
    % \documentclass[handout]{beamer}
    \documentclass[]{beamer}
    % ignorenonframetext
    \usetheme{Dresden}
\end{lstlisting}
\end{frame}

\begin{frame}[fragile]{Slides erstellen}
\begin{itemize}[<+->]
    \item ein Slide-Template
\end{itemize}
\begin{lstlisting}
    \begin{frame}[fragile]{TITEL}
    \begin{itemize}[<+->]
        \item 
    \end{itemize}
    \end{frame}
\end{lstlisting}
\end{frame}

%\title{Pause}

\title{Hausaufgaben}

\begin{frame}[fragile]{HA? HA!}
    \begin{itemize}[<+->]
    \item Hausaufgabe: versuchen Sie\footnote{Versuchen Sie es wirklich, das sind nur ein paar Zeilen, die anders sind!}, eines ihrer Dokumente unter \XeTeX zu erstellen
    \item Hausaufgabe 2: spielen Sie mit \lstinline|label| und \lstinline|ref| herum
    \item Hausaufgabe 2.5: machen Sie sich auch mit dem Paket \lstinline|varioref| vertraut, wo liegen hier die Unterschiede zu \lstinline|ref|?
    \item Hausaufgabe 3: seien Sie stark und inkludieren Sie alles einmal testweise!
    \item Hausaufgabe 4: setzen Sie den Zähler nach dem Deckblatt zurück
    \item Hausaufgabe 5: setzen Sie die Art der Seitenzahl der aktuellen Seite auf \texttt{roman}
    \end{itemize}
\end{frame}

\begin{frame}[fragile]{Ende}
    \begin{itemize}[<+->]
    \item damit wären wir auch schon wieder am Ende
    \item nun brauchen wir nur noch Fußnoten und ein Literaturverzeichnis
    \item und dann kommen die Matheumgebung und die Linguistikpakete!
    \item gut gemacht! Bis morgen!
    \end{itemize}
\end{frame}



\end{document}
