% Wir erstellen Tabellen
\section{Tabellen}
%\blindtext

\begin{table}[t!] 
    % t[op] b[ottom] h[ere] und andere? 
    % 'tbh' default, kombinierbar, präferenz ist abhängig von der Reihenfolge der Optionen 
    % '!' erzwingt: [h!]
    \begin{center}
        \begin{tabular}{|c|r|l|r|}\hline
            Pferde & Ziegen & Hasen & Schafen \\\hline\hline
            1 & 4 & 2 & 100 \\
            1 & 4 & 2 & 100 \\\hline
            1 & 4 & 2 & 100 \\\hline
            1 & 4 & 2 & 100 \\\hline
        \end{tabular}
    \caption[Tiere im Zoo. So.]{Das sind alle unsere Tiere im Zoo, aber eigentlich haben wir noch viel mehr, die wir gar nicht kennen und der Donald Knuth ist auch dabei als Stammgast.}
    \label{tab:zoo}
    \end{center}
\end{table}


Hallo meine Tiere in \ref{tab:zoo}
Das geht auch \vref{tab:zoo}


Hallo meine Tiere in \ref{tab:zoo}
Das geht auch \vref{tab:zoo}
%\blindtext
%\clearpage


% Wir erstellen Tabellen
\section{Tabellen mit Booktabs}
%\blindtext

\begin{table}[] 
    % t[op] b[ottom] h[ere] und andere? 
    % 'tbh' default, kombinierbar, präferenz ist abhängig von der Reihenfolge der Optionen 
    % '!' erzwingt: [h!]
    \begin{center}
        \begin{tabular}{c|r|l|r}\toprule
            Pferde & Ziegen & Hasen & Schafen \\\midrule
            1 & 4 & 2 & 100 \\
            1 & 4 & 2 & 100 \\
            1 & 4 & 2 & 100 \\
            1 & 4 & 2 & 100 \\\bottomrule
        \end{tabular}
        \caption[Tiere im Zoo. So.]{Das sind alle unsere Tiere im Zoo, aber eigentlich haben wir noch viel mehr, die wir gar nicht kennen und der Donald Knuth ist auch dabei als Stammgast.}
        \label{tab:booktabszoo}
    \end{center}
\end{table}

\clearpage
\listoftables