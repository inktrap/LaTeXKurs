\documentclass[]{beamer}
\usetheme{Berlin}

% unsere Pakete wie immer
\usepackage[T1]{fontenc}
\usepackage[ngerman]{babel}
% utf8x kann mehr sonderzeichen usw.
% wird aber wohl nicht mehr gepflegt und
% utf8 ist die bessere Wahl. Oder XeTeX
\usepackage[utf8]{inputenc}

\author{Der gesamte Kurs}
\title{Hello Kurs}
\date{24.12.2015}

% frame darf man NIE einrücken und es darf auch
% KEIN Kommentar nach dem Öffnen der Frame-Umgebung kommen!!
\begin{document}

\maketitle

\begin{frame}[fragile]{Ääähm Grundkurs Linguistik}
    \begin{itemize}
    \item Umgebung Frame aufmachen. Maaaaann. HAHAHAA.
    \end{itemize}
\end{frame}

% wir lassen items nach und nach erscheinen

\begin{frame}[fragile]{Ääähm Grundkurs Linguistik}
    \begin{itemize}[<+->]
        \item Umgebung Frame aufmachen.
        \item aber nach und nach
        \item Maaaaann. HAHAHAA.
    \end{itemize}
\end{frame}

% viele andere Optionen und Templates gibt es auch

\end{document}