% ein minimales XeTeX-Dokument
% siehe: <https://wiki.ubuntuusers.de/XeTeX#Bekannte-Probleme>
% für die Kompilierung muss man seine IDE bemühen
% gut ist auch <https://de.wikipedia.org/wiki/XeTeX>
\documentclass{article}
\usepackage{amsmath}
\usepackage{mathspec}
\usepackage{polyglossia}
\usepackage{xunicode}
\usepackage{xltxtra}
\usepackage{amsfonts,amsrefs,amsthm}

% language specific data
% you have to install the fonts,
% like <texlive-lang-japanese>
% like <texlive-lang-chinese>
\usepackage{xeCJK}

\begin{document}
    \section{Systemschrift sagt}
        电脑死机了。\\
        My computer has frozen.\\

    \section{Japanese}
        すべての人間は、生まれながらにして自由であり、
        かつ、尊厳と権利と について平等である。
        人間は、理性と良心とを授けられており、
        互いに同胞の精神をもって行動しなければならない。
\end{document}
