\begin{frame}[fragile]{Mathematik}
    \begin{itemize}[<+->]
        \item Achtung: eine kurze, unvollständige Einführung!!
        \item \texttt{amsmath}-Paket ist der Standard. \texttt{amssymb amsfonts} für Symbole, Fonts.
        \item Bereiche: im Text oder in einer Umgebung
        \item \texttt{inline}: \lstinline|\( Mathematikmodus \) $ Mathematikmodus $ |
        \item oder als Umgebung. Fragen: Wie?
        \item längere Formeln kann man als \texttt{displaymath} setzen:
        \item \lstinline|\[ \]| oder \lstinline|\begin{displaymath} \end{displaymath}|
        \item Gleichungen per \texttt{equation}-Umgebung
    \end{itemize}
\end{frame}

\begin{frame}[fragile]{Mathematik: Brüche, Wurzel}
    \begin{itemize}[<+->]
        \item Syntax: \lstinline|\frac{numerator}{denominator}|
        \item \lstinline|\(\frac{a}{b}\)|\is \(\frac{a}{b}\)
        \item Syntax: \lstinline|base^{exponent}| 
        \item Bsp: \lstinline|3^{20}|\is \(3^{20}\)
        \item \lstinline| \( 5 + 3^{2} = 14\)|\is \( 5 + 3^{2} = 14\)
        \item Syntax: \lstinline|\(\sqrt[root]{arg}\)| Bsp: \lstinline|\(\sqrt[3]{5}\)|\is \(\sqrt[3]{5}\)
        \item \lstinline|\(\sqrt{\frac{a}{b}}\)|\is \(\sqrt{\frac{a}{b}}\)
        \item Indices: \lstinline|\(a_{1}\)|\is \(a_{1}\)
    \end{itemize}
\end{frame}

\begin{frame}[fragile]{Beispiele}
    \begin{itemize}[<+->]
        \item \lstinline| \(\neg\forall x P(x) \equiv \exists x \neg P(x)\)|\is \(\neg\forall x P(x) \equiv \exists x \neg P(x)\)
        \item \lstinline| \( \frac{1}{n} \sum_{i=i}^{n} x_{i} \)|\is{} \( \frac{1}{n} \sum_{i=i}^{n} x_{i} \)
    \end{itemize}
    \begin{lstlisting}
\begin{displaymath}
    \cos A \cos B
    = \frac{1}{2}\left[ \cos(A-B)+\cos(A+B) \right]
\end{displaymath}
\end{lstlisting}
\begin{displaymath}\cos A \cos B
    = \frac{1}{2}\left[ \cos(A-B)+\cos(A+B) \right]
\end{displaymath}
\end{frame}

\begin{frame}[fragile]{Aligned numbered Equation}
    \begin{lstlisting}
    \begin{align}
    z_0 &= d = 0 \\
    z_{n+1} &= z_n^2+c
    \end{align}
    \end{lstlisting}
    \begin{align}
    z_0 &= d = 0 \\
    z_{n+1} &= z_n^2+c
    \end{align}
\end{frame}

\begin{frame}[fragile]{Theoreme, Beweise}
    \begin{itemize}[<+->]
        \item \lstinline| \begin{theorem} \end{theorem}| (\texttt{corollary lemma})
        \item Beispiele: \href{https://de.sharelatex.com/learn/Theorems_and_proofs}{theorems, definitions, corollaries and lemmas}
    \end{itemize}
\end{frame}


\begin{frame}[fragile]{Zusammenfassung}
    \begin{itemize}[<+->]
        \item Leerzeichen werden nicht interpretiert
        \item \texttt{\textbackslash} per Makro \lstinline|\backslash|
        \item \lstinline|^| und \lstinline|_| haben eine besondere Bedeutung
        \item es gibt ein paar Symbole: \href{https://en.wikibooks.org/wiki/LaTeX/Mathematics#List_of_Mathematical_Symbols}{per default}
        \item \href{http://milde.users.sourceforge.net/LUCR/Math/mathpackages/amssymb-symbols.pdf}{Amssymb hat mehr}
        \item Skoping ist wichtig: \lstinline|2^ab| ist nicht $ 2^{ab} $ sondern $ 2^ab$
        \item Bonus: \href{http://detexify.kirelabs.org/classify.html}{Symbolerkennung}
        \item Aufgabe: setzen Sie eine Formel für das arithmetische Mittel per Summenformel
    \end{itemize}
\end{frame}

