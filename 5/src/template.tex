\documentclass[]{beamer}
%\documentclass[handout]{beamer}
% ignorenonframetext
\usetheme{Dresden}
\setbeamertemplate{caption}[numbered]
\setbeamertemplate{caption label separator}{:}
\setbeamercolor{caption name}{fg=normal text.fg}
\usepackage{amssymb,amsmath}
\usepackage{textcomp}
\usepackage{ifxetex,ifluatex}
\usepackage{lmodern}
\usepackage{metalogo}
\usepackage{hyperref}
\usepackage{metalogo} % if you want to add LaTeX within your code
\usepackage{listings}
\usepackage{soul}
\usepackage{color}
\usepackage[T1]{tipa}

\ifxetex
  \usepackage{fontspec,xltxtra,xunicode}
  \defaultfontfeatures{Mapping=tex-text,Scale=MatchLowercase}
  \newcommand{\euro}{€}
\else
  \ifluatex
    \usepackage{fontspec}
    \defaultfontfeatures{Mapping=tex-text,Scale=MatchLowercase}
    \newcommand{\euro}{€}
  \else
    \usepackage[T1]{fontenc}
    \usepackage[ngerman]{babel}
    \usepackage[utf8]{inputenc}
      \fi
\fi
% use upquote if available, for straight quotes in verbatim environments
\IfFileExists{upquote.sty}{\usepackage{upquote}}{}
% use microtype if available
\IfFileExists{microtype.sty}{\usepackage{microtype}}{}

% Comment these out if you don't want a slide with just the
% part/section/subsection/subsubsection title:
\AtBeginPart{
  \let\insertpartnumber\relax
  \let\partname\relax
  \frame{\partpage}
}
\AtBeginSection{
  \let\insertsectionnumber\relax
  \let\sectionname\relax
  \frame{\sectionpage}
}
\AtBeginSubsection{
  \let\insertsubsectionnumber\relax
  \let\subsectionname\relax
  \frame{\subsectionpage}
}

\setlength{\parindent}{0pt}
\setlength{\parskip}{6pt plus 2pt minus 1pt}
\setlength{\emergencystretch}{3em}  % prevent overfull lines
\setcounter{secnumdepth}{0}

\lstset{
    language=[LaTeX]{TeX},
      basicstyle=\small\ttfamily,        % the size of the fonts that are used for the code
    }

\hypersetup{
     colorlinks   = true,
     linkcolor    = blue,
     urlcolor     = blue,
     citecolor    = gray
}

\newcommand{\is}{\textcolor{red}{:}\ }

\date{}
\author{Valentin Heinz}
\title{\LaTeX \ für Einsteiger}
\date{09.10.2015}
\subtitle{Fünfte Sitzung}

%\includeonly{}

\begin{document}

\maketitle

\title{Ablauf}
\begin{frame}[fragile]{Ablauf}
    \begin{itemize}[<+->]
        \item Hausaufgabenbesprechung
        \item \LaTeX\ ohne Mathe?
        \item Mathematik/Formeln
        \item Phonetik: IPA mit \texttt{tipa}
        \item Alinierungen, Beispiele, usw.: \texttt{covington}
        \item Zeichnen mit \texttt{tikz}
        \item Bäume: \texttt{forest}, \texttt{avm}
    \end{itemize}
\end{frame}


\title{Hausaufgaben}
\begin{frame}[fragile]{Hausaufgaben}

\begin{itemize}[<+->]
    \item Hausaufgabe: versuchen Sie\footnote{Das war doch schwieriger als gedacht!}, eines ihrer Dokumente unter \XeTeX zu erstellen
    \item Lösung: \href{https://github.com/inktrap/LaTeXKurs/tree/master/4/tex/}{XeTeX-Beispiel}
    \item Hausaufgabe 2: spielen Sie mit \lstinline|label| und \lstinline|ref| herum
    \item Lösung: \lstinline|\label{labelname}| und \lstinline|ref{labelname}| danach und mehrfach kompilieren
    \item Hausaufgabe 2.5: machen Sie sich auch mit dem Paket \lstinline|varioref| vertraut, wo liegen hier die Unterschiede zu \lstinline|ref|?
    \item \href{https://github.com/inktrap/LaTeXKurs/tree/master/4/tex/}{Beispiel}
\end{itemize}
    
\end{frame}

\begin{frame}[fragile]{Hausaufgaben}
    
    \begin{itemize}[<+->]
        \item Hausaufgabe 3: seien Sie stark und inkludieren Sie alles einmal testweise!
        \item dauert lange, oder?
        \item Hausaufgabe 4: setzen Sie den Zähler nach dem Deckblatt zurück
        \item Lösung: Seitenzähler zurücksetzen: \lstinline|\setcounter{page}{1}| oder \lstinline|\addtocounter{page}{-1}|
        \item Hausaufgabe 5: setzen Sie die Art der Seitenzahl der aktuellen Seite auf \texttt{roman}
        \item \begin{lstlisting}
        \pagenumbering{roman}
        foobar
        \clearpage
        \pagenumbering{arabic}
        foobar
        \end{lstlisting}
    \end{itemize}

\end{frame}



\title{Mathematik}
\section{test}

\begin{displaymath}
\neg \forall x P(x) \equiv \exists x \neg P(x)
\end{displaymath}

\begin{align}
z_0 &= d = 0 \\
z_{n+1} &= z_n^2+c
\end{align}

\begin{displaymath}
    \sum_{i}^{a}
\end{displaymath}


\title{Phonetik}
\begin{frame}[fragile]{Phonetik}
    \begin{itemize}[<+->]
        \item \texttt{tipa}-Paket: \href{ftp://ftp.dante.de/tex-archive/fonts/tipa/tipaman.pdf}{Anleitung}
        \item Vorteil: extrem kurze und kompakte Schreibweise von IPA-Zeichen und Akzenten, Diakritika usw.
        \item Optionen: \lstinline|\usepackage[T1]{tipa}|
        \item mehrere Umgebungen: \lstinline| \textipa \begin{IPA}\end{IPA}|
    \end{itemize}
\end{frame}

\begin{frame}[fragile]{Phonetik}
    \begin{quote}
        A shortcut character refers to a single character that is assigned to a specific phonetic symbol and that can be directly inputted by an ordinary keyboard.
    \end{quote}
    \begin{itemize}[<+->]
        \item Quelle und Beispiele, siehe Dokumentation
        \item Wiederholung: \href{https://www.internationalphoneticassociation.org/sites/default/files/IPA_chart_(C)2005.pdf}{IPA-Chart}
        \item \lstinline|\textipa{[""Ekspl@"neIS@n]}| \is \textipa{[""Ekspl@"neIS@n]}
        \item Diakritika und Akzente per Kurzform: \lstinline|\textsubstring{a}| als \lstinline|\r*a| ergibt \textipa{\r*a}
        \item Aufgabe: versuchen Sie die Trankription \textipa{"la:tE\c{c}} zu erstellen.
        \item \href{http://www.ling.ohio-state.edu/events/lcc/tutorials/tipachart/tipachart.pdf}{Hilfe?}
    \end{itemize}
\end{frame}

\title{Alinierung}
\begin{frame}[fragile]{Covington}
    \begin{itemize}[<+->]
        \item  \lstinline| \usepackage{covington} |
        \item sehr vielseitiges Paket, funktioniert aber nicht mit der Beamer-Klasse
        \item funktioniert auch nicht mit \texttt{tipa} und \texttt{gb4e}
        \item deshalb werden wir Minimalbeispiele für linguistische Pakete erstellen
        \item Covington-Funktionsumfang:
        \begin{itemize}
            \item Beispiele
            \item Sequenzalinierungen
            \item Merkmalsstrukturen
        \end{itemize}
    \end{itemize}
\end{frame}

\begin{frame}[fragile]{Covington: Alinierungen}
    \begin{itemize}[<+->]
        \item Alinierungen funktionieren, wenn die Anzahl der Wörter die selbe ist, findet die Zuordnung automatisch statt\footnote{ansonsten: \texttt{\{leicht anpassen\}}}:
    \end{itemize}
    \begin{lstlisting}
        \gll Dit is een Nederlands voorbeeld.
        This is a Dutch example.
        \glt 'This is an example in Dutch.'
        \glend
    \end{lstlisting}
    \begin{itemize}
        \item Aufgabe: alinieren Sie eigenständig zwei Sätze miteinander mit \texttt{covington}
    \end{itemize}
\end{frame}

\begin{frame}[fragile]{Covington: Merkmalsstrukturen}
    \begin{itemize}[<+->]
        \item die Beispiele sind der \href{ftp://ftp.dante.de/tex-archive/macros/latex/contrib/covington/covington.pdf}{Anleitung} entnommen
        \item \texttt{covington} kann auch Merkmalsstrukturen darstellen: \lstinline| \fs{case:nom \\ person:P} |
        \item oder:
    \end{itemize}
    \begin{lstlisting}
        \psr{\lfs{S}{tense:T}}
        {\lfs{NP}{case:nom \\  number:N}
            \lfs{VP}{tense:T \\ number:N} }
    \end{lstlisting}
\end{frame}

\begin{frame}[fragile]{Covington: Beispiele}
    \begin{itemize}[<+->]
        \item Umgebungen: \lstinline|\begin{example} \end{example}|
        \item \lstinline|\begin{examples} \end{examples}|
        \item automatische Nummerierung und typographische Anpassung
        \item Aufgabe: setzen Sie eigenständig zwei Beispiele mit \texttt{covington}
        \item weitere Beispiele für Beispielpakete: \texttt{linguex, gb4e, expex}
        \item probieren, probieren, probieren!
    \end{itemize}
\end{frame}

\title{Tikz}
\begin{frame}[fragile]{tikZ}
    \begin{itemize}[<+->]
        \item mit TikZ kann man zeichnen, malen usw.
        \item \texttt{tikz}-Paket laden.
        \item ein Bild: \lstinline|\begin{tikzpicture}\end{tikzpicture}|
        \item ein Quadrat erstellen: \lstinline| \path (0pt,0pt) -- (8pt,0pt) -- (8pt,9pt) -- cycle;|
        \item zeichnen mit \lstinline| \tikz \draw |
        \item Beispiele: \href{https://github.com/inktrap/LaTeXKurs/tree/master/5/tex/tikz}{kreis-dreieck-baum}
        \item tollere Beispiele \href{http://www.texample.net/tikz/examples/}{auf texample.net}
    \end{itemize}
\end{frame}


\title{Trees}
\begin{frame}[fragile]{Merkmalsstrukturen, Bäume}
    \begin{itemize}[<+->]
        \item Merkmalsstrukturen lassen sich mit dem Paket \href{http://www.essex.ac.uk/linguistics/external/clmt/latex4ling/avms/}{avm} darstellen
        \item \texttt{forest} kann benutzt werden um Bäume zu erstellen: \href{https://www.overleaf.com/3419620fkxwhd#/9651961/}{Beispiel}
    \end{itemize}
\end{frame}


\title{Hausaufgaben}

\begin{frame}[fragile]{HA? HA!}
    \begin{itemize}[<+->]
    \item Hausaufgabe: versuchen Sie\footnote{Versuchen Sie es wirklich, das sind nur ein paar Zeilen, die anders sind!}, eines ihrer Dokumente unter \XeTeX zu erstellen
    \item Hausaufgabe 2: spielen Sie mit \lstinline|label| und \lstinline|ref| herum
    \item Hausaufgabe 2.5: machen Sie sich auch mit dem Paket \lstinline|varioref| vertraut, wo liegen hier die Unterschiede zu \lstinline|ref|?
    \item Hausaufgabe 3: seien Sie stark und inkludieren Sie alles einmal testweise!
    \item Hausaufgabe 4: setzen Sie den Zähler nach dem Deckblatt zurück
    \item Hausaufgabe 5: setzen Sie die Art der Seitenzahl der aktuellen Seite auf \texttt{roman}
    \end{itemize}
\end{frame}

\begin{frame}[fragile]{Ende}
    \begin{itemize}[<+->]
    \item damit wären wir auch schon wieder am Ende
    \item nun brauchen wir nur noch Fußnoten und ein Literaturverzeichnis
    \item und dann kommen die Matheumgebung und die Linguistikpakete!
    \item gut gemacht! Bis morgen!
    \end{itemize}
\end{frame}



\end{document}
