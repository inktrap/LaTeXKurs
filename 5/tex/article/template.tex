\documentclass[12pt,oneside,paper=a4,ngerman]{scrartcl}

% -----------------------------
% Grundlagen, wichtige Pakete
% -----------------------------

\usepackage[T1]{fontenc}
\usepackage[ngerman]{babel}
% utf8x kann mehr sonderzeichen usw.
% wird aber wohl nicht mehr gepflegt und
% utf8 ist die bessere Wahl. Oder XeTeX
\usepackage[utf8]{inputenc}

\usepackage{blindtext}
\usepackage{newcent} % New Century Schoolbook Schriftart
\usepackage[german]{varioref} % gibt schönere Verweise, Namensschema beachten

\usepackage{booktabs} % schöne Tabellen
\usepackage{graphicx} % Bilder einbinden
\usepackage{hyperref}
% nachträgliche Konfiguration von hyperref
\hypersetup{
    colorlinks,
    linkcolor=black,
    citecolor=black,
    urlcolor=blue
}
% Typographie und Quotes
\usepackage{csquotes}

% -----------------------------
% Bibliographie
% -----------------------------

% Standardweg eine Bibliographie einzubinden
% dann stehen im Dokument die Befehle \autocite und \textcite
% zur verfügung, KEIN citet oder citep.
%\usepackage[backend=biber,bibstyle=authortitle]{biblatex}

% Ein anderer stil: biblatex-sp-undefined
% diesen hier verwenden wir, auch wenn es etwas komplexer ist
% hier stehen die Befehle citet und citep und cite wie gewohnt zur Verfügung
% bibliographie konfiguration von
% <https://github.com/semprag/biblatex-sp-unified>
% via biblatex
\usepackage[backend=biber,
    bibstyle=authoryear,
    bibstyle=biblatex-sp-unified,
    citestyle=sp-authoryear-comp,
    maxcitenames=3,
    maxbibnames=99]{biblatex}

% unsere bibliographie
\addbibresource{example.bib}

% -----------------------------
% Wissenschaftliche Pakete
% -----------------------------

% Mathe
\usepackage{amsmath} 
% Fonts
\usepackage{amsfonts} 
%Symbole
\usepackage{amssymb}
%Phonetics

% ACHTUNG!!
% linguistische Pakete werden isoliert getestet
%\usepackage[T1]{tipa}
% Beispiele, Alinierungen, Merkmalsstrukturen
%\usepackage{covington}

% -----------------------------
% Einstellungen für unser Dokument
% -----------------------------

\pagestyle{headings} % dies gibt uns eine Kopfzeile mit Titel und Nummer der Überschrift
%\pagenumbering{roman}
%\includeonly{linguistics}
%\includeonly{citations}

% -----------------------------
% unser eigentliches Dokument beginnt
% -----------------------------

\begin{document}
\begin{frame}{Titelseite}
\begin{itemize}[<+->]
    \item wir erstellen eine minimale Titelseite
    \item in unserer eigenen Datei, die wir einbinden
    \item und aktivieren !
    \item hierzu brauchen wir 4 Befehle
\end{itemize}
\end{frame}

\begin{frame}[fragile]{Inhaltsverzeichnis}
    \begin{itemize}[<+->]
        \item wir wollen auch ein Inhaltsverzeichnis
        \item dieses wird, wenn wir etwa \texttt{sections} haben, automatisch befüllt
        \item es wird mittels \lstinline|\tableofcontents| erstellt
        \item siehe auch: \lstinline|\listoftables| und \lstinline|\listoffigures|
        \item ziemlich cool, oder?
    \end{itemize}
\end{frame}


\begin{frame}{Struktur und Text}
    \begin{itemize}[<+->]
        \item Dokumenterstellung, Dokumentklasse, Optionen
        \item Sprachoptionen, Eingabeencoding, Buchstabenencoding
        \item Abschnitte und Unterabschnitte
        \item Blindtextpaket laden und Text einfügen
    \end{itemize}
\end{frame}

\begin{frame}{Wiederholung}
    \begin{itemize}[<+->]
        \item fontenc: Ausgabeformatierung (passende Ausgabe erzeugen) \textbf{zuerst laden!}
        \item babel: Sprachoptionen laden, z.B. Datumsanpassung an den Sprachraum
        \item inputenc: Eingabeformatierung (viele Zeichen/Buchstaben direkt eingeben) \textbf{danach laden!}
        \item zur Kompilierung benutzen wir \texttt{pdfTeX}
        \item aber es gibt auch \XeTeX und \LuaTeX
    \end{itemize}
\end{frame}

\begin{frame}[fragile]{Schriftformatierung}
    \begin{itemize}
        \item \href{https://tex.stackexchange.com/questions/59403/what-font-packages-are-installed-in-tex-live/59405#59405}{stackexchange} hilft weiter
        \item für helvetica einfach \lstinline|\usepackage{helvet}| verwenden. Fertig.
        \item einfach \lstinline|\usepackage{newcnt}| verwenden. Fertig.
        \item nicht-Blindtext kursiv, monospace, fett bzw. als Kapitälchen formatieren
        \item die vorherigen Formatierungen mischen
        \item Serifenformatierung und Romanische Schrift für ausgewählte Abschnitte benutzen
        \item \lstinline|\rmfamily{}| und \lstinline|\sffamily|
    \end{itemize}
\end{frame}

\begin{frame}{Ausrichtung, Absätze, Lass-mich-in-Ruhe}
    \begin{itemize}
        \item wie formatieren wir einen Teil rechtsbündig, Blocksatz bzw. linksbündig?
        \item wie erzeugen wir einen Zeilenumbruch?
        \item wie einen Paragraphen?
        \item wie unformatierten Text (Umgebung, inline)?
    \end{itemize}
\end{frame}




\thispagestyle{empty}
\blindtext
\clearpage
\section{Überschrift}
\blindtext
\clearpage
\thispagestyle{plain}
\blindtext
\clearpage
\section{Eine andere Überschrift}
\blindtext
\clearpage
\blindtext
\clearpage
\begin{frame}[fragile]{Tabellen}
    \begin{itemize}[<+->]
        \item eine Tabelle hat einen Anfang: \lstinline|\begin{tabular}|
        \item und ein Ende \lstinline|\end{tabular}|
        \item dann kommt eine Kopfzeile: \lstinline| Pferde & Schafe & Ziegen & Hasen \\|
        \item dann eben Zeilen mit Werten: \lstinline| 1 & 2 & 3 & 4 \\ |
        \item jetzt wissen wir auch warum wir \& maskieren müssen.
    \end{itemize}
\end{frame}

\begin{frame}[fragile]{Tabellen mit Strichen}
    \begin{itemize}[<+->]
        \item einen vertikalen Strich fügen wir mit \lstinline|\hline| hinter \lstinline|\\| hinzu
        \item die Zellenausrichtung bestimmen wir vorher \lstinline|\begin{tabular}{lrlr}|
        \item möglich ist auch \texttt{c}. Wichtig: Pro Spalte eine Angabe, maximal!
        \item was passiert bei weniger/mehr?
        \item was macht \lstinline!\begin{tabular}{l|r|l|r}|!?
    \end{itemize}
\end{frame}

\begin{frame}[fragile]{table-Umgebung}
    wir legen unsere Tabelle (\texttt{tabular}) in eine \texttt{table}-Umgebung
        \begin{lstlisting}
        \begin{table}[h!]
            \begin{center}
                \caption{Caption for the table.}
                \label{tab:table1}
                % tabelle hier hin
            \end{center}
        \end{table}
        \end{lstlisting}
\end{frame}

\begin{frame}[fragile]{Neu gelernt!}
    \begin{itemize}[<+->]
        \item \lstinline|\table|
        \item Platzierung ist wichtig!
        \item Tabellenplatzierung \lstinline|\begin{table}[Position] ... \end{table}|
        \item Parameter: \lstinline|b, h, p und t (Default: tbp)| (Kombination mit \texttt{!} erzwingt!)
        \item \lstinline|\centering|
        \item \lstinline|\caption|
        \item \lstinline|\label| und \lstinline|\ref|? 
    \end{itemize}
\end{frame}
\section{Grafiken einbinden}

% das hier ist eine willkürliche Skalierung
%\includegraphics[scale=0.1]{../img/cat.jpg}
% wenn eine Grafik im selben Verzeichnis liegt,
% dann muss man nur den Namen angeben: cat (ohne Dateiendung)
%\includegraphics[]{cat}

% Nachteil: unübersichtlich bei mehreren Grafiken

% das hier ist nicht so schön, weil verzerrt
%\includegraphics[width=1cm, height=5cm]{../img/cat.jpg}

% textweite, texthöhe
% ist uns aber zu viel zu hoch
%\includegraphics[width=\textwidth, height=\textheight]{../img/cat.jpg}

% textweite, texthöhe
% TODO: ist uns aber zu viel zu hoch, deshalb verkleinern wir
%\includegraphics[width=\textwidth, height=\textheight]{../img/cat.jpg}

% dynamische Größenanpassung mit beibehaltenen Verhältnissen
%\includegraphics[width=0.8\textwidth,
%height=0.2\textheight,
%keepaspectratio]
%{../img/cat.jpg}

%\begin{figure}
%\end{figure}

\section{Katzen mit Labeln}

\begin{figure}[h!] % figure mit platzierungsoptionen
    \begin{center}
\includegraphics[width=0.8\textwidth,
    height=0.2\textheight,
    keepaspectratio]
    {../img/cat.jpg}
    \caption{Zwei flauschige Katzen.}
    \label{fig:cat}
    \end{center}
\end{figure}

\begin{figure}[h!] % figure mit platzierungsoptionen
    \begin{flushleft}
        \includegraphics[width=0.8\textwidth,
        height=0.2\textheight,
        keepaspectratio]
        {../img/cat.jpg}
        \caption{Zwei flauschige Katzen.}
        \label{fig:cat-left}
    \end{flushleft}
\end{figure}

\begin{figure}[h!] % figure mit platzierungsoptionen
    \begin{flushright}
        \includegraphics[width=0.8\textwidth,
        height=0.2\textheight,
        keepaspectratio]
        {../img/cat.jpg}
        \caption[Katzen-rechts]{Zwei flauschige Katzen.}
        \label{fig:cat-right}
    \end{flushright}
\end{figure}

Katzen wie in \ref{fig:cat} sind flauschig.
Katzen wie in \ref{fig:cat-left} sind links und flauschig.
Katzen wie in \ref{fig:cat-right} sind rechts und flauschig.


\subsection{Optionen}
\begin{itemize}
     \item \texttt{scale=WERT}: skaliert eine Grafik, Angabe von \texttt{WERT} ist sinnvoll und schön im (geschlossenen) Intervall $[0,1]$ aber es gehen auch beliebige Werte. Andere Angaben sind auch möglich.
     \item wir können auch die Breite und die Höhe bestimmen: \texttt{width=WERT} und \texttt{height=WERT}.
     \item \texttt{WERT} kann aber auch die Ausgabe eines Markos sein.
\end{itemize}

% viel mehr: <https://en.wikibooks.org/wiki/LaTeX/Floats,_Figures_and_Captions>


\clearpage
\listoffigures

\section{test}

\begin{displaymath}
\neg \forall x P(x) \equiv \exists x \neg P(x)
\end{displaymath}

\begin{align}
z_0 &= d = 0 \\
z_{n+1} &= z_n^2+c
\end{align}

\begin{displaymath}
    \sum_{i}^{a}
\end{displaymath}

Hallo ich bin Text.

% das sind die Standardzitierbefehle unter biblatex,
% sie können verwendet werden, wenn der
% Standardweg eine Bibliographie einzubinden gewählt wurde.
%Und das ist ein Zitat: \textcite[12]{Knuth:1997:ACP:260999}.\\
%Und das ist ein Zitat: \autocite[12]{Knuth:1997:ACP:260999}.\\

% das sie hier sind die stile wie wir so gewohnt
% sind, von z.B. bibtex oder von biblatex-sp-undefined
% dafür muss der stil biblatex-sp-undefined gewählt sein.

Und das ist ein Zitat: \citet[12]{Knuth:1997:ACP:260999}.
\begin{quote}
    \blindtext
\end{quote}
Und das ist ein Zitat: \citep[12]{Knuth:1997:ACP:260999}.
\begin{quote}
    \blindtext
\end{quote}

\printbibliography
\end{document}
